\conclusion
Основные результаты исследований, проведённых в диссертации, можно
сформулировать следующим образом:
\begin{list}{\labelitemi}{\leftmargin=1em}
\item Проведён полный анализ $dp$-взаимодействий, полученных с помощью
  100-сантиметровой водородной пузырьковой камеры ЛВЭ ОИЯИ при импульсе
  3.35~ГэВ/с. Оценены потери в упругом рассеянии и получены значения поперечных
  сечений отдельных каналов $dp$-взаимодей\-ствий. Качество полученной
  экспериментальной информации свидетельствует о её пригодности для последующего
  физического анализа.

  \vspace{1ex}
  Впервые в эксклюзивной постановке исследована безмезонная реакция \dpfrag.
  Определено дифференциальное сечение реакции перезарядки
  $(d\sigma/dt)_{\dpchex}\,|\,_{t=0} = 30.2 \pm 4.1$~мб/(ГэВ/с)$^{2}$.

  \vspace{1ex}
  Впервые в пучке дейтронов получено отношение $R_{\np}$ дифференциальных
  сечений перезарядки при $t=0$ в реакции \dpchex и \np. В рамках импульсного
  приближения полученное значение $R_{\np} = 0.55\,\pm\,0.08$ свидетельствует о
  преобладающем вкладе спин-зависящей части сечения \np рассеяния и согласуется
  с данными других экспериментов в области близких энергий.

\item На основе результатов исследований с помощью водородной пузырьковой камеры
  был предложен электронный эксперимент для изучения реакции перезарядки на
  дейтроне в области энергий выше 1~ГэВ. Рассмотрена возможность определения
  спин-зависящей части амплитуды элементарной перезарядки \np на основе прямого
  измерения дифференциального сечения $(d\sigma/dt)_{\dpchex}$ при $t=0$.

  \vspace{1ex}
  При активном участии диссертанта была создана установка СТРЕЛА, основными
  элементами которой являются блоки дрейфовых камер. Впервые проведено
  облучение установки в пучке дейтронов импульса 3.5~ГэВ/с на ускорительном
  комплексе Нуклотрона ЛФВЭ ОИЯИ. Использование быстрой современной электроники
  повышает эффективность набора данных.

  \vspace{1ex}
  В процессе создания и усовершенствования установки СТРЕЛА были разработаны и
  реализованы комплексы программ обработки и анализа экспериментальных
  данных. Полученное значение пространственного разрешения дрейфовых
  камер лежит в диапазоне 90--120~мкм, что позволяет осуществить исследования
  зарядово-обменных процессов во взаимодействиях дейтронов с протонами.
\end{list}

\vfill
Работа выполнена в Лаборатории физики высоких энергий ОИЯИ им.~В.И.~Векслера и
А.М.~Балдина. Автор выражает благодарность дирекции лаборатории за
предоставленную возможность проведения исследований.
Автор считает своим приятным долгом выразить глубокую благодарность научному
руководителю В.В.~Глаголеву за постановку темы диссертационной работы, за особо
внимательное отношение к моей работе на всех её этапах и многочисленные ценные
дискуссии и советы. Искренне благодарен Г.~Мартинской и Н.М.~Пискунову за
большое внимание к выполненной работе, за многократные научные обсуждения
вопросов, за поправки и конструктивную критику. Автор выражает благодарность за
обсуждение теоретических вопросов Н.Б.~Ладыгиной.

Я признателен сотрудникам лаборатории, оказавшим помощь и всестороннюю поддержку
в сотрудничестве с которыми выполнена эта работа: С.Н.~Базылеву, Ю.П.~Бушуеву,
Д.А.~Кириллову, А.А.~Повторейко, В.М.~Слепнёву и И.В.~Слепнёву.
Автор глубоко признателен всем, чья помощь и поддержка сделала возможным
появления данного труда.

%%% Local Variables:
%%% mode: latex
%%% TeX-master: "musinsky_disser"
%%% coding: utf-8
%%% End:
