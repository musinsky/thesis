\intro
В последние годы, вместе с развитием техники эксперимента, появилась реальная
возможность осуществления исследований, использующих поляризованные пучки и
мишени в ускорительных центрах мира. Возникла так называемая спиновая физика,
задачами которой стали решения классических задач ядерной и субъядерной физики,
интерес к которым давно возбуждался теоретиками.

Состояния систем тождественных частиц описываются волновыми функциями, которые
являются либо симметричными, либо антисимметричными при перестановке любой пары
частиц. В квантовой механике доказывается, что симметрия функций системы частиц
остаётся неизменной в любой момент времени. Частицы обладающие полуцелым спином
и подчиняющиеся статистике Ферми"--~Дирака, описываемые антисимметричными
волновыми функциями, подчиняются принципу Паули, запрещающему в данной системе в
одном и том же квантовом состоянии находиться более чем одному фермиону. В
данной работе рассматривается простейшая система двух нуклонов~--- дейтрон. В
соответствии с выше сказанным, волновая функция этой системы обладает свойствами
антисимметрии.

Дейтрон является слабосвязанной системой из нейтрона и протона. Спин дейтрона
равен единице, энергия связи 2.23 МэВ. Вследствие малой энергии связи, импульсы
внутриядерного Ферми движения нуклонов также являются небольшими (порядка 50
МэВ/с) и при больших энергиях падающих на дейтрон частиц слабо влияют на
квазинуклонное взаимодействие с протоном или нейтроном ядра. Это позволяет
применять к рассмотрению взаимодействий на дейтроне так называемое импульсное
приближение.

При взаимодействиях высокоэнергичных частиц с дейтроном заметно выделяется класс
вторичных частиц, называемых спектаторами, относительно которых принимается, что
они являются как бы <<наблюдателями>> при квазинуклонном столкновении падающей
частицы с <<другим>> нуклоном дейтрона. На практике импульсные распределения
спектаторов хорошо описываются с помощью известных волновых функций дейтрона,
что подтверждает применимость импульсного приближения. В случае малых
передаваемых импульсов при взаимодействии разумно предполагать, что квантовые
состояния спектаторов и нуклонов отдачи сохраняются такими же, какими они были
в составе дейтрона. Это особенно отчётливо проявляется в случае обратной
кинематики, когда ядро (в данном случае дейтрон) падает на протон.

В теории нуклон-нуклонного рассеяния, фундаментальное значение имеет извлечение
(восстановление) комплексных амплитуд матрицы рассеяния. Для получения всех
амплитуд необходимо проводить так называемый полный опыт, в результате которого
должен быть получен такой набор экспериментальных наблюдаемых, который позволяет
провести исчерпывающее описание процесса. Полный эксперимент включает в себя
измерения с поляризованными частицами-снарядами, а также с поляризованными
мишенями. Это весьма большая и трудоёмкая задача.

Однако, в некоторых случаях возможно определить отдельные амплитуды матрицы
рассеяния, либо их совокупность, путём выбора некоторых экспериментальных
условий. Одной из возможностей является изучение реакции перезарядки на
дейтроне, которая \! при некоторых условиях определяется только зависящими от
спина компонентами амплитуд. Таких ограничений не возникает для перезарядки на
свободном нуклоне. Эта идея была формализована математически в целом ряде
теоретических работ. Важно то, что появилась возможность извлечь спин-зависящую
часть амплитуды $np$-рассеяния с использованием неполяризованных протонов и
неполяризованных дейтронов в реакциях перезарядки на дейтроне. Экспериментально
такую серию работ пытались решать в основном, в пучках сепарированных нейтронов
и, в последствии, в квази-монохроматических нейтронах от стриппинга ускоренных
дейтронов.

Мы сочли целесообразным провести анализ \! экспериментальных данных, в которых
ускоренные дейтроны падали на протонную мишень. При такой постановке опыта
дейтроны монохроматичны, а вторичные два протона~--- продукты перезарядки
дейтрона на протоне, являются быстрыми и вылетают в переднем направлении под
малыми углами. Такой эксперимент был проведён на синхрофазотроне ЛВЭ ОИЯИ с
использованием в качестве одновременно детектора и мишени водородной пузырьковой
камеры. До начала наших исследований другие эксперименты с пучком дейтронов
практически отсутствовали.

Анализ экспериментальных данных оправдал наши надежды и кроме того, по мере
развития детектирующей техники и систем сбора данных, позволил сформулировать
требования к решению задачи электронными методами. Была создана установка
СТРЕЛА, которая начала функционировать на пучке дейтронов, ускоренных на
Нуклотроне ЛФВЭ ОИЯИ.

Из вышесказанного видно, что тема диссертации является актуальной. Целью
диссертационной работы является:
\begin{itemize}
\item Проведение полного анализа реакции перезарядки в дейтрон-протонных
  взаимодействиях на водородной пузырьковой камере для определения
  спин-зависящей части амплитуды $np$-рассеяния.
\item Предложение на основе этого анализа постановки эксперимента с
  использованием современной электронной методики и обоснование выбора геометрии
  эксперимента.
\item Испытание всех элементов установки СТРЕЛА, в том числе дрейфовых камер в
  качестве трековых приборов высокой разрешающей способности.
\item Создание систем программного обеспечения во время облучения установки и
  восстановления треков в дрейфовых камерах.
\end{itemize}

В настоящей работе приведены результаты физических и методических исследований
по теме диссертации. Она состоит из введения, четырёх глав, заключения и списка
цитируемой литературы. Во введении кратко обсуждаются предпосылки постановки
задачи.

Первая глава диссертации посвящена истории вопроса, теоретическому формализму
основной идеи эксперимента, а также результатам исследования элементарной
реакции \np перезарядки в различных экспериментах.

Вторая глава содержит результаты анализа реакции перезарядки в дейтрон-протонных
взаимодействиях полученных на жидководородной пузырьковой камере. С помощью
измерения дифференциального поперечного сечения реакции перезарядки \dpchex
оценён вклад спин-зависящей амплитуды \np рассеяния.

В третьей главе сформулировано предложение нового электронного эксперимента и
подробно описывается установка СТРЕЛА. Обсуждается структура установки и её
основные элементы. Внимание уделено дрейфовым камерам, являющимся основными
составляющими установки.

Четвёртая глава посвящена процедуре восстановления треков и соответствующему
программному обеспечению. В заключении сформулированы основные результаты
диссертационной работы. Список цитируемой литературы включает ссылки на
92 работу по теме диссертации.

Анализ экспериментальных данных и результаты методических исследований,
включённые в диссертацию, опубликованы в виде статей в научной периодике, а
также докладывались на ряде международных конференций и семинаров. В том числе,
на международных совещаниях «Релятивистская ядерная физика: от сотен МэВ до ТэВ»
в Стара Лесна (Словакия~--- 2000 и 2009~г.) и Модра Гармония (Словакия~---
2006~г.), на рабочем совещании по протон-дейтронным взаимодействиям (Дубна~---
2002~г.), на международном семинаре по проблемам физики высоких энергий
<<Релятивистская ядерная физика и квантовая хромодинамика>> (Дубна~--- 2000,
2002 и 2006~г.) и на семинарах ЛФВЭ ОИЯИ. Список публикаций, на которых основана
диссертация, содержит 12 наименований.

%%% Local Variables:
%%% mode: latex
%%% TeX-master: "musinsky_disser"
%%% coding: utf-8
%%% End:
