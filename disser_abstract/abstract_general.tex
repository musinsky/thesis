\clearpage
\setcounter{page}{1}
\begin{center}
  {\large\textbf{Общая характеристика работы}}
\end{center}
\noindent \textbf{Актуальность работы}

В теории нуклон-нуклонного рассеяния фундаментальное значение \linebreak
имеет извлечение (восстановление) комплексных амплитуд матрицы рассеяния. Для
получения всех амплитуд необходимо проводить так называемый полный опыт, в
результате которого должен быть получен такой набор экспериментальных
наблюдаемых, который позволяет провести исчерпывающее описание процесса. Полный
эксперимент включает в себя измерения с поляризованными частицами-снарядами, а
также с поляризованными мишенями. Это весьма большая и трудоёмкая задача.

Однако, в некоторых случаях возможно определить отдельные амплитуды матрицы
рассеяния, либо их совокупность, путём выбора специальных экспериментальных
условий. Одной из таких возможностей является изучение реакции перезарядки на
дейтроне, которая при некоторых допущениях определяется только зависящими от
спина компонентами амплитуд нейтрон-протонного упругого рассеяния. Эта идея была
формализована математически в целом ряде теоретических работ. Важно то, что
появилась возможность извлечь спин-зависящую часть амплитуды $np$-рассеяния с
использованием неполяризованных протонов и неполяризованных дейтронов в реакциях
перезарядки на дейтроне. Экспериментально такую серию работ пытались выполнить
на пучках сепарированных или квази-монохроматических нейтронов от стриппинга
ускоренных дейтронов.

На синхрофазотроне ЛВЭ ОИЯИ был проведён эксперимент с использованием водородной
пузырьковой камеры, в котором ускоренные дейтроны падали на протонную мишень.
При такой постановке опыта \mbox{дейтроны} монохроматичны, а два вторичных
протона (продукты реакции перезарядки дейтрона на протоне) являются быстрыми и
вылетают в переднем направлении под малыми углами. До начала наших исследований
другие эксперименты с пучком дейтронов практически отсутствовали.

Анализ экспериментальных данных оправдал ожидания, и по мере развития
детектирующей техники и систем сбора данных, позволил сформулировать требования
к решению задачи электронными методами. Была создана установка СТРЕЛА, которая
начала функционировать на пучке дейтронов, ускоренных на Нуклотроне ЛФВЭ ОИЯИ.

\vspace{2ex}
\noindent \textbf{Цель диссертационной работы}
\begin{list}{\labelitemi}{\leftmargin=1.25em}
\item Проведение анализа реакции перезарядки \dpchex в дейтрон-протонных
  взаимодействиях на водородной пузырьковой камере для определения
  спин-зависящей части амплитуды \np рассеяния.
\item Предложение электронного эксперимента (установка СТРЕЛА) с целью
  исследования зарядово-обменных процессов на пучке дейтронов. Испытание
  элементов установки, в том числе дрейфовых камер в качестве трековых приборов
  высокой разрешающей способности.
\item Создание систем программного обеспечения для работы во время облучения
  установки СТРЕЛА на пучке ускоренных дейтронов и восстановления треков в
  дрейфовых камерах.
\end{list}

\vspace{2ex}
\noindent \textbf{Научная новизна}
\begin{list}{\labelitemi}{\leftmargin=1.25em}
\item Проведён анализ $dp$-взаимодействий, полученных с помощью
  100-сантиметровой водородной пузырьковой камеры ЛВЭ ОИЯИ при импульсе
  3.35~ГэВ/с. Оценены потери в упругом рассеянии и получены значения поперечных
  сечений отдельных каналов $dp$-взаимодействий. Качество экспериментальной
  информации свидетельствует о её пригодности для последующего физического
  анализа.
\item Впервые в эксклюзивной постановке исследована безмезонная реакция \dpfrag.
  Определено дифференциальное сечение реакции перезарядки
  $(d\sigma/dt)_{\dpchex}\,|\,_{t=0} = 30.2 \pm 4.1$~мб/(ГэВ/с)$^{2}$.
\item Впервые на пучке дейтронов получено отношение $R_{\np}$ дифференциальных
  сечений перезарядки при $t=0$ в реакциях \dpchex и \np. В рамках импульсного
  приближения полученное значение $R_{\np} = 0.55\,\pm\,0.08$ свидетельствует о
  преобладающем вкладе спин-зависящей части сечения \np рассеяния и согласуется
  с данными других экспериментов в области близких энергий.
\item При использовании результатов исследований, полученных на водородной
  пузырьковой камере, был подготовлен проект электронного эксперимента для
  изучения реакции перезарядки на дейтроне в области энергий выше 1~ГэВ.
  Рассмотрена возможность определения спин-зависящей части амплитуды
  элементарной перезарядки \np на основе прямого измерения дифференциального
  сечения $(d\sigma/dt)_{\dpchex}$ при $t=0$.
\item При активном участии диссертанта была создана установка СТРЕЛА~---
  одноплечевой магнитный спектрометр, основными элементами которой являются
  блоки дрейфовых камер в качестве координатных детекторов. Впервые проведено
  облучение установки в пучке дейтронов импульса 3.5~ГэВ/с на ускорительном
  комплексе Нуклотрона ЛФВЭ ОИЯИ.
\item В процессе создания и усовершенствования установки СТРЕЛА были разработаны
  и реализованы комплексы программ обработки и анализа экспериментальных данных.
  Полученное значение пространственного разрешения дрейфовых камер лежит в
  диапазоне 90--120~мкм, что позволяет осуществить исследования
  зарядово-обменных процессов во взаимодействиях дейтронов с протонами.
\end{list}

\vspace{2ex}
\noindent \textbf{Практическая значимость}

Выполненная работа является частью программы исследований малонуклонных систем
в рамках проблемно-тематического плана ОИЯИ. Использование слабосвязанной
системы двух нуклонов, какой является дейтрон, в совокупности с его квантовыми
числами, позволяет в рамках простых предположений (импульсное приближение)
с применением принципа Паули получить сведения о вкладе спин-зависящей части
амплитуды \np рассеяния, не прибегая к сложным поляризационным исследованиям.
При таком подходе достаточно изучить дифференциальные поперечные сечения
зарядово-обменных процессов в $dp$ и $np$-взаимодействиях при малых переданных
импульсах. Полученные результаты могут быть использованы для фазового анализа,
а при продвижении в область более высоких энергий, позволяют проверить и
усовершенствовать имеющиеся модели сильного взаимодействия нуклонов.

Важным является совершенствование экспериментальной установки путём применения
современных методов исследований и соответствующего программного обеспечения.
Полученные результаты могут быть использованы при планировании новых
экспериментов, направленных на углублённое исследование взаимодействий
элементарных частиц и атомных ядер.

\vspace{2ex}
\noindent \textbf{На защиту выносятся следующие основные результаты}
\begin{list}{\labelitemi}{\leftmargin=1.25em}
\item Результаты анализа экспериментальных данных по взаимодействиям дейтронов
  импульса 3.35~ГэВ/с с протонами, полученных на водородной пузырьковой камере.
\item Результаты изучения реакции \dpfrag, выделение канала перезарядки \dpchex
  и получение дифференциального сечения этого канала при нулевом переданном
  импульсе.
\item Оценка вклада спин-зависящей части амплитуды $np$-рассеяния из процесса
  перезарядки на дейтроне в рамках импульсного приближения.
\item Результаты методических исследований, направленных на развитие установки
  СТРЕЛА на Нуклотроне, и создание программного обеспечения, необходимого для
  анализа полученных экспериментальных данных.
\end{list}

\vspace{2ex}
\noindent \textbf{Апробация работы и публикации}

Результаты исследований, включённых в диссертацию, опубликованы в виде статей в
научной периодике, а также докладывались на ряде международных конференций и
семинаров. В том числе, на международных совещаниях «Релятивистская ядерная
физика: от сотен МэВ до ТэВ» в Стара Лесна (Словакия~--- 2000 и 2009~г.) и Модра
Гармония (Словакия~--- 2006~г.), на рабочем совещании по протон-дейтронным
взаимодействиям (Дубна~--- 2002~г.), на международном семинаре по проблемам
физики высоких энергий <<Релятивистская ядерная физика и квантовая
хромодинамика>> (Дубна~--- 2000, 2002 и 2006~г.) и семинарах ЛФВЭ ОИЯИ.

Список публикаций, содержащих основные результаты диссертации, приведён в конце
автореферата~\cite{m01,m02,m03,m04,m05,m06,m07,m08,m09,m10,m11,m12}.

\vspace{2ex}
\noindent \textbf{Структура и объём диссертации}

Диссертация состоит из введения, четырёх глав и заключения. Общий объём
составляет 112 страниц машинописного текста, включая список литературы из 92
наименований.

%%% Local Variables:
%%% mode: latex
%%% TeX-master: "musinsky_abstract"
%%% coding: utf-8
%%% End:
