\institution{
  \vspace{-15ex}
  ОБЪЕДИНЁННЫЙ ИНСТИТУТ ЯДЕРНЫХ ИССЛЕДОВАНИЙ}

\vspace*{10ex}
\def\asmanuscript{}
\begin{flushright}
  На правах рукописи \\
  УДК~539.172.13
\end{flushright}

\author{
  \vspace{7ex}
  МУШИНСКИ Ян}
\topic{Исследование зарядово-обменных процессов \\
  в дейтрон-протонных взаимодействиях}
\specnum{01.04.16}
\spec{Физика атомного ядра и элементарных частиц}
\title{АВТОРЕФЕРАТ \\ диссертации на соискание учёной степени \\
  кандидата физико-математических наук}
\city{Дубна}
\date{2010 \vspace{3ex}}
\maketitle

\thispagestyle{empty}
\begin{center}
  Работа выполнена в Лаборатории физики высоких энергий \\
  Объединённого института ядерных исследований.
\end{center}

\vskip 5ex \noindent
\begin{tabularx}{\textwidth}{lp{1em}X}
  Научный руководитель:
  & & доктор физико-математических наук, \\
  & & профессор \\
  & & Глаголев Виктор Викторович \\ \\

  Официальные оппоненты:
  & & доктор физико-математических наук \\
  & & Литвиненко Анатолий Григорьевич \\ \\

  & & доктор физико-математических наук \\
  & & Сокол Гарри Арсентьевич \\ \\

  Ведущая организация:
  & & Научно-исследовательский институт \\
  & & ядерной физики МГУ, г.~Москва
\end{tabularx}

\vskip 10ex \noindent
Защита состоится
<<\,\rule[0pt]{2em}{0.25pt}\,>>~\rule[0pt]{6em}{0.25pt}~2010~г.
в~\rule[0pt]{3em}{0.25pt}~часов на заседании диссертационного совета
Д~720.001.02 при Лаборатории физики высоких энергий им.~В.И.~Векслера и
А.М.~Балдина Объединённого института ядерных исследований, 141980, г.~Дубна,
Московская область.

\vskip 10ex \noindent
С диссертацией можно ознакомиться в библиотеке ЛФВЭ ОИЯИ.

\vskip 3ex \noindent
Автореферат разослан
<<\,\rule[0pt]{2em}{0.25pt}\,>>~\rule[0pt]{6em}{0.25pt}~2010~г.

\vfill \noindent
Учёный секретарь диссертационного совета \\
кандидат физико-математических наук \hfill Арефьев В.А.

%%% Local Variables:
%%% mode: latex
%%% TeX-master: "musinsky_abstract"
%%% coding: utf-8
%%% End:
