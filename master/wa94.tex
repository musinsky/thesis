\chapter{Experiment WA94}
Experiment WA94, podobne ako aj WA85, bol navrhnutý za účelom skúmania 
produkcie podivných a multipodivných častíc v~oblastiach centrálnych rapidít
a priečnych hybností väčších $p_{T}>0.6\: GeV/c$ v~zrážkach ťažkých
iónov používajúc CERN Omega spektrometer. V~roku 1991 prebiehal
experiment na SPS (Super Proton Synchrotron), kde sa urýchľoval zväzok
iónov $^{32}S$ na energiu 200 GeV/c na nukleón a so sírovým terčíkom (na
rozdiel od WA85, kde sa použil wolfrám ako terčík). V~roku 1993 sa použil
200 GeV zväzok protónov na porovnanie s~experimentálnymi údajmi z~roku
1991. Experiment prebiehal vo West Experiment Area v~CERNe, preto skratka WA
v názve experimentu.

Na štúdium produkcie podivnosti v~zrážkach ťažkých iónov boli
vyvinuté dve metódy [5]. Jeden prístup je pokúsiť sa merať čo najväčší
počet častíc. Výhodou je získanie kompletnej informácie o~danom prípade.
Nevýhodou je prítomnosť veľkého pozadia, tak ako ukazuje obr 3.1 interakciu
sírového zväzku pri energii 200 GeV/c na nukleón so sírovým terčíkom
v~experimente NA35.


\begin{center}
  \includegraphics*[bb=177 600 403 743,height=6cm]{zrazka.eps}
\end{center}
\begin{center}
  Obrázok 3.1: Interakcia S-S v~experimente NA35.
\end{center}
\newpage
Na rekonštrukciu rozpadu danej častice je potrebný pár opačne nabitých dráh
rozpadových produktov. Pri danom prístupe sa počet všetkých dráh
zaznamenaných na prípad (event) rapídne zväčšuje a hľadanie konkrétnych
dráh je o~to ťažšie. Okrem toho efektívnosť rekonštrukcie dráhy bude klesať,
ako napr. v~mnohovláknových proporcionálnych komorách, kde efektívnosť je
nepriamo úmerná k~počtu dráh. Taktiež  je
nezanedbateľný čas potrebný na analýzu každého prípadu, ktorý sa predlžuje s
jeho zložitosťou.

Druhý prístup, ktorý sa použil v~experimentoch WA85 aj WA94, je použitie
špeciálneho experimentálneho nastavenia na štúdium hadrónov v~úzkom fázovom
priestore. Zaznamenávame menej informácií o prípade, pozadie je značne menšie
ako v~prvom prístupe. Efektívnosť rekonštrukcie je pre tieto dráhy tiež
vyššia. Tento efekt sa dosiahol použitím tzv. \uv{butterfly} geometrie v
experimentoch WA85 a WA94.

\section{\uv{Butterfly} geometria}
Schopnosť tejto metódy redukovať pozadie je možné pochopiť, keď si
predstavíme dve dráhy častíc pochádzajúcich z~terčíka. Nech tieto
častice majú rovnaké priečne hybnosti $p_{T}$, ale rozdielne pozdĺžne
hybnosti $p_{L}$. V~danej vzdialenosti od terčíka častice s~rovnakým
$p_{T}$, ale menším  $p_{L}$ budú vytvárať v~rovine y-z priemety
väčších kružníc, B=0. V~prítomnosti magnetického poľa sú dráhy častíc
vychýlené na jednu stranu. Dráhy častíc s~menším  $p_{L}$ (majú väčšie
priemety kružníc) sú vychyľované viac ako dráhy s~väčším  $p_{L}$ 
(menšie kružnice). Ak sa berú do úvahy všetky hodnoty  $p_{L}$ pre dané 
$p_{T}$, ukazuje sa, že miesta dráh tvoria dve oblasti tvaru V, ako
ukazuje obr. 3.2.

Človek s~predstavivosťou môže vidieť v~tvare vytieňovaných oblastí podobnosť
s~krídlami motýľa, preto usporiadanie  je známe ako \uv{butterfly} geometria.
Mnohovláknové proporcionálne komory použité vo WA94 sú modifikované práve
touto geometriou, avšak je používané len horné krídlo. 

Aktívne plochy komôr boli vybrané tak, aby boli schopné detegovať častice 
s~priečnymi hybnosťami $p_{T}>0.6\: GeV/c$. Podivné hadróny s~nižšími $p_{T}$
vznikli vo fireballe pravdepodobne skôr, ako mohli vzniknúť 
extrémne podmienky na vznik plazmy, a preto nás nezaujímajú a nebudeme nimi
zbytočne zaťažovať detektor~[5]. Asi $30\%$ častíc s~$p_{T}>0.6\: GeV/c$
prejde priamo cez citlivú oblasť komôr.


\newpage
\vspace*{1cm}
\begin{center}
  \includegraphics*[bb=155 345 500 765]{butterfly.eps}
\end{center}
\begin{center}
  Obrázok 3.2: Princíp \uv{butterfly} geometrie.
\end{center}

\newpage
\section{Experimentálne usporiadanie}
Usporiadanie experimentálnej aparatúry ukazuje obr. 3.3. Zväzok iónov
vchádzajúci do spektrometra Omega je dodávaný zo SPS.
Tento  ultrarelativistický zväzok 
iónov síry je fokusovaný dipólovými a kvadrupólovými magnetmi.
\subsection{Spektrometer Omega}
Spektrometer ${\Omega}$ je viacúčelové zariadenie umiestnené vo West Area v
CERNe.  Pozostáva  z~páru supravodivých Helmholtzových cievok, ktoré generujú
homogénne magnetické pole o~veľkosti 1.8 Tesla. V~tomto poli dlhom 6m je
umiestnený terčík a rôzne detektory.  

V~rune z roku 1991 bol použitý tenký terčík $^{32}S$ o~hrúbke $800 \mu m$ (2
\% interakčnej dĺžky). Tento terčík bol umiestnený vo vzdialenosti
$x_{\Omega}=-215 \: cm$, pričom x-ová os je totožná so smerom zväzku
častíc, poloha $x_{\Omega}=0 \: cm$ zodpovedá stredu Omega magnetu (viď obr.
3.3).

Multiplicita sekundárnych nabitých častíc sa merala pomocou systému
silikónových mikrostrípových komôr umiestnených pri terčíku. 
Pozície terčíka a siedmych mnohovláknových proporcionálnych komôr (MWPC) sú
iné ako v experimente WA85, aby sa dosiahla väčšia  akceptancia
MWPC.

\subsection{Silikónové mikrostrípové komory}
Na určovanie multiplicity nabitých častíc slúžia dva rady silikónových
mikrostrípových  komôr vzdialených 15 cm od terčíka v~smere zväzku.
Priestor pokrytý týmito mikrostrípmi je v~oblasti v~intervale pseudorapidít 
$2.2 < \eta < 3.5$. 
% Pseudo\-rapidita je definovaná ako
% $\eta=-\ln(\tan(\theta/2))$ a v~relativistickej limite je ekvivalentná
% laboratornej rapidite $y_{LAB}$ . 
Tak sa registrujú práve tie častice,
ktoré vznikli v~tejto centrálnej oblasti v~danom intervale pseudorapidít,
kde sa predpokladá formovanie QGP [5].

\newpage
\begin{center}
\includegraphics*[bb=80 130 500 735,height=20.5cm]{aparatura.eps}
\end{center}
\begin{center}
  Obrázok 3.3: Usporiadanie experimentu WA94.
\end{center}

\newpage
 
\begin{center}
\includegraphics*[bb=130 485 515 740]{yuvcham.eps}
\end{center}
\begin{center}
  Obrázok 3.4: Schématické znázornenie MWPC.
\end{center}

\subsection{Mohovláknové proporcionálne komory (MWPC)}
Na detekciu dráh sekundárnych častíc slúži sedem
MWPC. Každá z~týchto komôr pozostáva z~troch rovín vodivých vlákien (U,V,Y),
kde vlákna Y roviny sú vertikálne. Vlákna U~a V~rovín
týchto komôr su vychýlené o~$+10.14^{\circ}$ a $-10.14^{\circ}$ od vertikály.
%a tak nám poskytujú informácie o~$z_{\Omega}$-ovej a $y_{\Omega}$-ovej
%zložke súradnice. 
Každá rovina obsahuje 752 vodivých (anódových) vlákien  vzdialených od seba
2mm. Vzdialenosť medzi každým anódovým vláknom a katódovou rovinou je $8\:
mm$. Ako plyn sa používa zmes argónu, isobutánu, freónu a etanolu. Komory su
očíslované od A1 do A7, komora najbližšia k~terčíku A1 je vzdialená od stredu
magnetu $x_{\Omega}=71\: cm$, ostatných šesť komôr má súradnice
$x_{\Omega}=90, 109, 147, 190, 215$ a $250\: cm$. 

\uv{Butterfly} efekt sa dosiahol modifikovaním katódovej roviny, ktorá
pozostá\-va z~$12\: \mu m$ mylárovej fólie pokrytej grafitovým náterom.
Odstránením časti grafitového náteru z~katódovej roviny vzniká elektricky
izolovaná časť tvaru \uv{butter\-fly}. Na obr. 3.4 môžeme vidieť, aká časť
grafitu bola odstránená. Aktívna časť komory je udržiavaná na pracovnom
potenciáli veľkosti okolo 5 kV, zatiaľ čo neaktívna oblasť má potenciál 
o~0.5 kV menší. 

\newpage
\subsection{Experimentálne údaje}
Získané experimentálne údaje (raw data)  boli zaznamenané na
OMEGA online VAX  v~EPIO formáte (strojovo nezávislý formát používaný
vo~fyzikálnych experimentoch). Eventy sú zapisované v~16 bit-ových slovách
typu \uv{integer čísiel} s~tzv. header blokom pozostávajúcim z  20
slov, ktorý obsahuje informácie o~danom evente a jeho dĺžke. Informácie
o~signáli z~vlákna, veľkosti clustra atď. sú obsiahnuté v~ROMULUS bloku.
Tieto dáta sú potom spracované pomocou programu TRIDENT  [17], ktorý je
modifikovaný pre usporiadanie WA94 experimentu. Pre každé experimentálne
usporiadanie sa pripravila nová verzia programu. TRIDENT rekonštruuje najprv
dráhu v~Y rovine, ak ju zrekonštruuje, tak potom k~nej hľadá príslušné
dráhy aj v~U~a V~rovine. Táto metóda hľadania dráh je omnoho rýchlejšia ako
vytvárať všetky možné kombinácie hitov z~Y,U a V~rovín súčasne. 

