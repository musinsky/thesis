%\documentstyle[graphicx,a4,12pt,emlines,bezier,slovak]{report}

%\begin{document}
\chapter{Výpočet akceptancie a efektívnosti rekonštrukcie}
V~predchádzajúcej kapitole bolo ukázané, ako sú rekonštruované podivné častice
(konkrétne $K^{0}_{S}$ mezóny). Avšak surový počet rekonštruovaných častíc
nám  sám o~sebe hovorí veľmi málo. Pre štúdium QGP vytvorenej zrážkou
ultrarelativistických iónov je okrem iného potrebný aj priemerný počet
každého druhu častíc produkovaných v~interakciách a nie holý počet
rekonštruovaných častíc. To nám umožní porovnanie produkcie rôzných druhov
podivných častíc. Napríklad porovnanie výťažkov multipodivných baryónov 
s~podivnými baryónmi nám poskytne informáciu o~obsahu podivnosti vo
vzniknutom  fireballe. Túto informáciu môžu teoretici využiť na dedukciu
najviac pravdepodobného modelu (hadrónový plyn alebo QGP), ktorý najlepšie
popisuje dané výsledky.

Na počítanie skutočných výťažkov podivných častíc  a tak porovnanie
relatívnych produkčných výťažkov je potrebné, aby bol počet rekonštruovaných
častíc korigovaný na:
\begin{itemize}
  \item{Geometrickú akceptanciu detektora.}
  \item{Efektívnosť rekonštrukcie a analyzačného softwaru.}
\end{itemize}
Geometrická akceptancia detektora je v~podstate meranie, ako geometria
experimentálnej aparatúry obmedzuje počet registrovaných častíc. Dráhy
častíc, ktoré boli v~experimente detegované, vychádzali z~terčíka v~rozmedzí
úzkeho priestorového uhla, približne 2\% z~celkového  $4\pi$. Ak by detektory
pokrývali celý priestorový uhol veľkosti $4\pi$, tak potom by nimi prešli
všetky vzniknuté častice a akceptancia by bola 100\%. Samozrejme, že
komplexná povaha $K^{0}$ rozpadu v~porovnaní s~dráhami z~terčíka komplikuje
pochopenie, ako geometria detektora ovplyvňuje rekonštrukciu takýchto
rozpadov, ale robí to nesmierne dôležitým. Priestorové obrezania (cuts),
ktoré sme prediskutovali pre rozpady $K^{0}$ mezónov v~tretej kapitole -
dráhy z~rozpadového produktu musia prejsť cez určený počet komôr,
alebo že x-ová súradnica vrcholu musí ležať v~určitej oblasti atď.,
nevyhnutne ovplyvnia akceptanciu.

Pod efektívnosťou rekonštrukcie sa rozumie schopnosť  použitého softwaru
správne rekonštruovať signál z~detektora. Efektívnosti rekonštrukcie
sú determinované schopnosťou  rekonštrukčného softwaru  z~hľadiska nájdenia
dráh a zisťovania rozpadového vrcholu. Do počítania efektívnosti
rekonštrukcie je tiež zahrnutá  elektronická efektívnosť dôležitých častí
detektora použitého v~experimente, v~našom prípade mnohovláknové
proporcionálne komory. Tá samozrejme znižuje počet priestorových bodov 
vytvorených na jednu dráhu a tak nepriaznivo ovplyvňuje hľadanie dráh a
v~konečnom dôsledku aj identifikáciu vrcholov.

\section{Akceptancia $K^{0}$ mezónu}
Na určenie geometrickej akceptancie detektora je generovaný známy počet Monte
Carlo častíc v~širšej oblasti fázoveho priestoru ako pokrývajú
jednotlivé komory. Oblasť $y-p_{T}$ sa rozdelí do mriežky a akceptancia
sa počíta pre každý element mriežky osobitne. Táto mriežka je rozdelená na
elementy rozmeru $0.1\: GeV/c$ v~$p_{T}$ a $0.05$ v~rapidite. Použitím
mnohých takýchto štvorcov sme vytvorili oblasť, ktorá bude citlivá na
postupné zmeny v~akceptancii v~celom študovanom rozmedzí $y-p_{T}$.

Častice sú generované do hornej polovice $(z>0)$ Omega magnetu, keďže
dolná oblasť nebola v~experimente využitá.  Každá častica sa rozpadá
izotrópne s~dĺžkou rozpadu danou 
strednou dobou života $\tau$ $K^{0}$ mezónu. Pravdepodobnosť $P(x_{0})$, že
častica o~hmotnosti $M$ prejde vzdialenosť $x_{0}$  alebo väčšiu je daná
vzťahom
\begin{center}
  $P(x_{0})=e^{-Mc^{2}x_{0}\Gamma/|p|}$,
\end{center}
kde $\Gamma = 1 / c\tau$ a $p$ je hybnosť častice. Dosadením do
predchádzajúcej rovnice dostávame vzdialenosť rozpadu
\begin{center}
  $x_{0}=-\frac{c\tau |p|}{M_{K^{0}}}\ln(R)$,
\end{center}
kde $R$ je náhodné číslo generované v~intervale 0 až 1 a $c\tau$ je pre
$K^{0}$ rovné $2.6762~cm$. Po vypočítaní rozpadovej vzdialenosti sa zistí
pozícia vrcholu $K^{0}$ a tá musí byť v~intervale $-100<x_{\Omega}<71\: cm$,
keďže toto je oblasť, v~ktorej ležia rekonštruované $K^{0}$ vrcholy.
Na vrchol sa tiež kladie už spomínaná podmienka Armenterovského --
Podolanského $\alpha$, ktoré leží v~intervale $(+0.45<\alpha<-0.45)$.
 
Rozpadové produkty  sa potom nechajú prechádzať cez nasimulované vláknové
komory. Zisťuje sa počet tých, ktoré prešli cez všetkých sedem komôr  ako
príslušné reálne dráhy.
Akceptancia je potom definovaná ako podiel počtu zistených a generovaných
častíc
\begin{center}
  $akceptancia=\frac{pocet.zistenych}{pocet.generovanych} \times \frac{1}{2}
\times 0.686$.
\end{center}
Faktor 1/2  berie do úvahy, že častice boli generované iba do hornej
polovice magnetu Omega a faktor 0.686 vyjadruje  \uv{branching ratio} 
$K^{0}_{S} \longrightarrow \pi^{+} \pi^{-}$, t.j. podiel tohto kanálu k
celkovému počtu možných rozpadov  $K^{0}$ mezónu, ktoré v~experimente nie je
možné priamo pozorovať. Pre každý bod  $y-p_{T}$ mriežky bolo generovaných 
1 000 000 $K^{0}$ mezónov. Výslednú závislosť akceptancie od rapidity a
priečnej hybnosti môžeme vidieť na obr. 5.1.

\begin{center}
  \includegraphics*[width=14.4cm]{accept.eps}
\end{center}
\begin{center}
 Obrázok 5.1: Akceptancia $K^{0}$ mezónu ako funkcia rapidity a $p_{T}$.
\end{center}

\newpage
\section{Efektívnosť rekonštrukcie}
Cieľom experimentu WA94
je študovať relatívne výťažky podivných a multipodivných častíc a ich
antičastíc. Preto okrem počítania akceptancie detektorov je potrebné do
Monte Carlo prípadov zahrnúť aj fyzikálnu účinnosť detektorov
 a efektívnosť softwaru, ktorý rekonštruuje dráhy a
určuje rozpadové vrcholy.  

Za týmto účelom bolo generovaných 20 000 Monte Carlo častíc. Častice
sú  generované v~terčíku a potom sa nechajú rozpadať tak, že rozdelenie
rozpadovej dĺžky zodpovedá tomu, čo sa predpokladá pre
dobu života častice. Rozpadové produkty sa potom  nechajú prejsť cez Omega
magnetické pole, pritom sa  hľadajú body, v~ktorých pretínajú každú rovinu
komory. V~každom bode pretnutia sa generuje cluster vláknových zásahov
(hitov) berúc do úvahy fyzikálnu efektívnosť jednotlivých komôr. Tieto
generované prípady sa potom zmiešajú s~reálnymi prípadmi z experimentu. Monte
Carlo hity potom splynú s reálnymi prípadmi. Takéto prípady sú potom
v~podstate nerozlíšiteľné od prípadov obsahujúcich skutočný rozpad. Takto
novo vytvorený súbor dát je  potom ďalej spracovaný rovnakými programami, aké
sa používajú pre reálne dáta.  Pri rozpoznaní Monte Carlo častíc, ktoré sú
úspešne rekonštrované, potom možno zistiť efektívnosť rekonštrukcie 
napríklad ako  funkciu multiplicity  častíc v dráhových komorách. 
Multiplicita MWPC je daná podielom celkového počtu clustrov v každej rovine a
počtom rovín. Vývojový diagram na obr.~5.2
sumarizuje proces použitý na výpočet efektívnosti rekonštrukcie.

Efektívnosť rekonštrukcie, ako už bolo spomenuté vyššie, závisí od dvoch
faktorov - efektívnosti vláknových komôr pri zaznamenávaní dráhy spôsobenej
prechodom častice cez komory a efektívnosti softwaru pri správnom hľadaní
dráh a rozpadových vrcholov.

\subsection{Počítanie efektívnosti komôr}
Na vypočítanie efektívnosti komory je najskôr spustený program TRIDENT na
vzorke  dát. Pritom študovaná komora je vynechaná z~procesu zostavovania
dráh, čím sú rekonštruované dráhy úplne nezávislé od tejto komory. Program
na počítanie efektívnosti komôr, ktorý číta TRIDENT výstup, potom počíta
zrekonštruované \uv{dobré} dráhy, ktoré prešli cez každú rovinu komory a
overí sa, či každé vlákno do $2\: mm$ (rozostupenie vlákien) od nárazu
vzplanie. Ak áno, potom je táto rovina považovaná za úspešne detegujúcu.
\uv{Dobrá} dráha pre účel počítania efektívnosti komôr je tá, ktorá 
prechádza cez A1-A7 komory, zanecháva  šesť priestorových bodov - možné
maximum a smeruje späť do terčíka. Efektívnosť roviny je potom podiel
počtu  úspešne detegovaných dráh a celkového počtu dráh. Tento postup sa
opakuje pre  každú zo siedmych komôr.

\newpage
\vspace*{3cm}
\hspace*{-0.3cm}
  \input{diag.pic} 
\vspace*{2cm}
\begin{center}
  Obrázok 5.2: Procesy použité pri výpočte efektívnosti rekonštrukcie.
\end{center}
\newpage

Za účelom štúdia efektívnosti komôr je každá rovina rozdelená na štyri
logické kvadranty, ako vidno na obr. 5.3 a pre každý kvadrant sa počíta
osbitne. Keďže komora pozostáva z~troch rovín, počíta sa 84 efektívnosti
(7 komôr $\times$ 3 roviny $\times$ 4 kvadranty). Tie sa potom
použijú pri implantácii Monte Carlo dát do reálnych prípadov.

\begin{center}
  \includegraphics*[bb=65 275 530 465,width=14.4cm]{4caskom.eps}
\end{center}
\begin{center}
  Obrázok 5.3: Štyri logické kvadranty MWPC.
\end{center}

Dalo sa očakávať, že efektívnosť vláknových komôr na okraji butterfly
oblasti (viď časť 3.2.3) bude klesať. Skúmalo sa, či efektívnosť komôr bude
k~okraju klesať postupne, alebo či existuje nejaký ostrý skok. Je to
dôležité vedieť, keďže postupné klesanie efektívnosti by bolo zdrojom
systematickej chyby.  Ukázalo sa, že efektívnosť postupne klesá k~nule na
okraji veľkom $\sim$ 1 cm. Tiež sa ukázalo, že počet dráh klesá smerom 
k~okraju, keďže aj počet vhodných zrekonštruovaných bodov je v~tejto oblasti
príliš malý. Preto bolo rozhodnuté vylúčiť  okrajový pás veľkosti 1 cm. Tým
sa zaistilo, že analýza bola vykonávaná v~oblasti rovnomernej efektívnosti.

Efektívnosť komôr použitých v~roku 1991 pri interakciách síra-síra bola
vypočítaná na vzorke dát. Primerná efektívnosť všetkých siedmych A~komôr  
ležala v intervale 83\% - 95\% [5].
% \newpage
% \begin{center}
% \hspace*{-1.2cm}
% \includegraphics*[bb=274 512 510 754,width=7cm]{efekom1.eps}
% \end{center}
% \begin{center}
%   Obrázok 5.4: Efektívnosť roviny U~komory A1.
% \end{center}
% \begin{center}
% \hspace*{-0.7cm}
% \includegraphics*[bb=106 280 478 645,width=12cm]{efekom2.eps}
% \end{center}
% \begin{center}
%   Obrrazok 5.5: Efektívnosť všetkých siedmich A~komôr.
% \end{center}

\subsection{Štruktúra \uv{raw} dát}
Aby Monte Carlo dáta reprodukovali reálne dáta, musí byť zložený event
vytvorený splynutím generovaných a reálnych, v~presne rovnakom formáte ako
raw dáta. Preto je dôležité pochopiť štruktúru raw eventu, vstup pre
TRIDENT. Ako bolo spomenuté na konci tretej kapitoly, raw dáta sú zapísané na
pásku  v~EPIO (strojovo nezávislom) formáte. Event je napísaný vo forme 16
bitových slov a každý event má logickú štruktúru (obr. 5.4).
Začiatočný (header) blok s~NH úvodnými slovami dáva informácie o~evente,
jeho dĺžku, typ atď. Najdôležitejšie slovo je prvé, ktoré udáva dĺžku
záznamu. Použitím tohto akoby ukazovateľa je program schopný pozerať na
koniec záznamu, ktorý je \uv{top} koncom ROMULUS štruktúry. Tento ROMULUS
blok obsahuje všetky informácie o~vláknových hitoch, veľkosti clusterov atď.,
ktoré TRIDENT potrebuje na rekonštrukciu dráh v~komorách. 
\begin{center}
  \includegraphics*[bb=97 575 505 770]{sctraw.eps}
\end{center}
\begin{center}
  Obrázok 5.4: Štruktúra raw eventu.
\end{center}
TRIDENT potom dáva výstupný záznam pozostávajúci z~viacerých blokov, s~prvým
(header) blokom, ktorý obsahuje informácie o~štruktúre záznamu a je na obr.
5.5.  Elektronický dátový blok je práve blok ROMULUS, ktorý bol
vstupom pre TRIDENT. Výsledky rekonštrukcie dráh sú dané v~geometrickom
bloku.

\subsection{Implantácia Monte Carlo častíc}
Na implantáciu Monte Carlo dráhy prechádzajúcej cez Omega magnetické pole do
ROMULUS bloku reálneho eventu, mixovací program najskôr sleduje dráhu cez
všetkých sedem komôr a hľadá body pretnutia s~každou vláknovou rovinou. Ak
dráha prechádza medzi dvoma vláknami, obidve vlákna vzplanú práve vtedy, ak
je vzdialenosť dráhy od vlákna väčšia ako 1/3 z~2 mm rozostupenia, inak
vzplanie vlákno, ktoré je bližšie k~dráhe.

Aby sa zaistilo, že sa Monte Carlo dráhy správajú ako reálne dráhy, berú
sa  v~úvahu kvadranty efektívnosti  komôr, ktorými dráhy prechádzajú pred
tým, ako sa generuje hit. Tým je zabezpečené, že generované častice nie
sú nevyhnutne dokonale detegované všetkými vláknovými komorami.

Takéto Monte Carlo dáta sú potom zlúčené s~reálnymi dátami do nového ROMULUS
bloku s~upraveným header blokom, ktorý berie do úvahy vznik ďalších slov. 
Celý takýto event je potom zapísaný na pásku v~EPIO formáte a môže prejsť cez
reťazec spracovateľských a analyzačných programov, práve tak, akoby to bol
reálny event. Posledná úloha je potom overiť, v~koľkých zo spracovaných
zložených eventov možno nájsť implantované častice.
\begin{center}
  \includegraphics*[bb=122 572 470 770]{scttrid.eps}
\end{center}
\begin{center}
  Obrázok 5.5: Štruktúra TRIDENT header bloku.
\end{center}


\subsection{Efektívnosť rekonštrukcie $K^{0}$}
Ako už bolo spomínané v~kapitole 4, vyžadujeme, aby $K^{0}$ kandidáti ležali
v~intervale $-100<x_{\Omega}<71$ cm a boli jednoznačné, t.j. aby premenná
$\alpha$  Podolanského -- Armenterosa bola  $-0.45<\alpha<+0.45$.
Okrem toho požadujeme, aby $K^{0}$ mali rapiditu a $p_{T}$ v~intervale
$2.5<y_{lab}<3.3$ a $1<p_{T}<3\: GeV/c$, keďže mimo tejto oblasti
akceptancia pre $K^{0}$ klesá
pod $\sim$ 10\% jej maximálnej hodnoty [5].

Avšak Monte Carlo častice musia byť generované v~trocha väčšom
akceptančnom okne, aby sa brali do úvahy nepresnosti v~meraní hybnosti a
polohy v~dôsledku konečnej citlivosti detekčného systému. Ak by boli Monte
Carlo častice generované v rovnakých oblastiach, v akých boli použité
finálne obmedzenia, niektoré by sa stratili, pretože by sa dostali mimo
uvažovanú oblasť. Pre tento prípad bolo generovaných 20 000 $K^{0}$  mezónov
s nasledujúcimi charakteristikami:
\begin{itemize}
 \item vrchol leží v intervale $-135<x_{\Omega}<71$ cm,
 \item rovnomerné rozdelenie rapidity na intervale $2.3<y_{lab}<3.5$,
 \item $0.8<p_{T}<3.2$ GeV/c,
 \item distribúcia $m_{T}$
\begin{center}
$\frac{dN}{dm_{T}} \sim m_{T}^{\frac{3}{2}}\exp^{-5 m_{T}}$,
\end{center}
 \item Monte Carlo rozpady sú izotropné.

\end{itemize}
Generované prípady potom zmiešame s reálnymi prípadmi. Takýto zmiešaný
prípad je potom ďalej spracovaný rovnakým reťazcom programov, ako sa
použil pre reálne dáta, t.j. TRIDENT, STRIPV0 a analyzačný program, ktorý
aplikuje rovnaké obrezania (cuts) ako pre reálne rozpady
\begin{itemize}
  \item vrchol leží v intervale $-100<x_{\Omega}<71$ cm,
  \item rovnomerné rozdelenie rapidity na intervale $2.5<y_{lab}<3.3$,
  \item $1<p_{T}<3$ GeV/c,
  \item hybnosť $K^{0}$ je $>6.5$ GeV/c,
  \item $-0.45<\alpha<+0.45$.
\end{itemize}
Efektívnosť rekonštrukcie je potom daná 
\begin{center}
  Efektívnosť=$\frac{N_{rek}}{N_{gen}}$,
\end{center}
kde $N_{gen}$ je počet generovaných prípadov, ktoré vyhovujú akceptančným
podmienkam a $N_{rek}$ je počet úspešne rekonštruovaných prípadov.
Na obr. 5.6 a 5.7 môžeme vidieť závislosť efektívnosti rekonštrukcie
$K^{0}_{S}$ od priemerného počtu clustrov v rovine pre magnetické pole hore a
dole.

\newpage
\begin{center}
  \includegraphics*[bb=3 3 560 560,width=8cm]{efeup.eps}
\end{center}
\begin{center}
  Obrázok 5.6: Efektívnosť ako funkcia multiplicity MWPC
clustrov pre pole hore.
\end{center}

\begin{center}
  \includegraphics*[bb=3 3 560 560,width=8cm]{efedo.eps}
\end{center}
\begin{center}
  Obrázok 5.7: Efektívnosť ako funkcia multiplicity MWPC
clustrov pre pole dole.
\end{center}




%\end{document}
