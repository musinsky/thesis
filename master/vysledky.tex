%\documentstyle[a4,12pt,slovak,graphicx]{report}

%\begin{document}

\chapter{Výsledky spracovania experimentálnych údajov}
Cieľom tejto diplomovej práce bolo skúmať produkciu $K^{0}_{S}$ mezónov v
experimente WA94 v zrážkach jadier S-S . Hlavným zdrojom
informácií boli získané  rozdelenia  invariantných hmotností, rapidít,
priečnych hybností a priečnych hmotností. Každá z týchto charakteristík
predstavuje informáciu o niektorej stránke fyzikálneho procesu zrážky.
Priečna hmotnosť môže súvisieť s teplotou zdroja [18], rapidita a priečna
hybnosť nám poskytne informácie o centralite zrážky a rozdelenie invariantnej
hmotnosti nám hovorí o kvalite rekonštrukcie študovanej častice na základe
použitých kritérií. Okrem  experimentu WA94 sa štúdiom kvark-gluónovej plazmy
zaoberali alebo zaoberajú aj experimenty ako napr. NA34, NA35, NA36, NA38,
NA44, NA45, WA80, WA85, WA97, E802, E810, E814.

Analyzačný program číta údaje z  DST pások, ktoré boli získané
spracovaním experimentálnych údajov programami TRIDENT a STRIPV0.
DST páska obsahuje  1 390 246 prípadov (eventov).
Po spracovaní analyzačným programom, kde sa použili konečné obrezania
(cuts) a napĺňali histogrami s horeuvedenými fyzikálnymi charakteristikami,
nám vyšiel výsledný počet  jednoznačných  $K^{0}_{S}$ mezónov na 17 067.

Hlavná zložka magnetického poľa $B_{z}$ bola v smere osi z \uv{pole hore} a
oproti smeru osi z \uv{pole dole}. Experimentálne údaje sa z metodických
dôvodov rozdelili na dve časti, jedna pre pole hore a druhá pre
pole  dole, aby sa mohli prejaviť prípadné  hrubé systematické chyby,
spojené s rôznymi orientáciami magnetického poľa v spektrometri Omega.

Na obr. 6.1 sú zobrazené nekorigované a) a korigované b) spektrá
invariantných  hmotností $K^{0}_{S}$ mezónov. Získané rozdelenie je
nesymetrické s maximom v \uv{bine} so stredom 0.495 GeV. Stredná hodnota
efektívnych hmotností pre obidve spektrá je 0.482 GeV, čo je o niečo menej
ako hodnota uvádzaná v tabuľkách, 0.497 GeV (Particle Physics Booklet).
Stredná hodnota a poloha maxima boli rovnaké aj pre pole hore a pre pole
dole, na obrázku je rozdelenie hmotností pre pole hore. Ako vidíme použité
kritéria rozumne reprodukujú hmotnosť $K^{0}_{S}$. Korigované spektrá nie sú
fitované žiadnou funkciou, pretože sa ukázalo, že ani fitovanie Gaussovou ani
Breit--Wignerovou krivkou nedáva dobrý vysledok [18].
\begin{center}
  \includegraphics*[bb=2 3 560 560,width=7cm]{emh.eps}
  \includegraphics*[bb=2 3 560 560,width=7cm]{emhcok.eps}
\end{center}
\begin{center}
  Obrázok 6.1: Spektrum invariantnej hmotnosti $K^{0}_{S}$ mezónov.
\end{center}

Rozdelenie priečnych hybností $K^{0}_{S}$ mezónov vidíme na obr. 6.2 a 6.3.
Na  obr. 6.2  je rozdelenie nekorigovaných a) a korigovaných (na
akceptanciu a efektívnosť rekonštrukcie) b) priečnych hybností pre
smer magnetického poľa hore. Na obr. 6.3 pre smer magnetického poľa dole.
Na obr. 6.4 je rozdelenie korigovanej rapidity pre pole hore a) a pre pole
dole b).  Ako môžeme vidieť aj z týchto histogramov, rozdiel výsledkov pre
smer magnetického poľa hore a dole je minimálny, preto  ďalej uvádzané
rozdelenia ostatných charakteristík sú sumárne  pre pole hore a
dole.  Tiež je vidieť, že korekcie majú  vplyv na početnosti v
jednotlivých binoch a menia tvary rozdelení očakávaným spôsobom (symetrizujú
rozdelenie invariantných hmotností a spektrá $p_{T}$ sú bližšie k
exponenciálnemu tvaru).

% korigované spektrá neboli fitované, možno povedať, že častokrát uvádzaný
% predpoklad o exponenciálnom a rovnomernom rozdelení $p_{T}$ a $y$, je
% splnený prinajmenšom kvalitatívne.

\newpage
\begin{center}
  \includegraphics*[bb=2 3 560 560,width=7cm]{pt.eps}
  \includegraphics*[bb=2 3 560 560,width=7cm]{ptc.eps}
\end{center}
\begin{center}
  Obrázok 6.2: Rozdelenie $p_{T}$ hybností pre magnetické pole hore.
\end{center}
\begin{center}
  \includegraphics*[bb=2 3 560 560,width=7cm]{ptd.eps}
  \includegraphics*[bb=2 3 560 560,width=7cm]{ptdowc.eps}
\end{center}
\begin{center}
  Obrázok 6.3: Rozdelenie $p_{T}$ hybností pre magnetické pole dole.
\end{center}

\begin{center}
  \includegraphics*[bb=2 3 560 560,width=7cm]{ych.eps}
  \includegraphics*[bb=2 3 560 560,width=7cm]{ycd.eps}

\end{center}
\begin{center}
  Obrázok 6.4: Rozdelenie rapidity $K^{0}_{S}$ mezónov.
\end{center}


\section{Rozdelenie $m_{T}$}
Skúmaním  priečnych hmotností  môžeme získať  informácie o
veličine, ktorá je v určitých modeloch interpretovaná ako teplota
prostredia, v ktorom sa formovali častice [18]. Pre takéto častice danej
rýchlosti je predpovedané  rozdelenie priečnych hmotností  tvaru
\begin{center}
  $\frac{1}{m_{T}}\frac{dN}{dm_{T}}\sim\beta m_{T} K_{1} (\beta m_{T}),$
\end{center}
kde $K_{1}$ je modifikovaná Besselova funkcia [19], $\beta$ je sklon, ktorého
prevrátená hodnota  sa zvykne v literatúre označovať ako  teplota.  Dve bežne
používané priblíženia sú
\begin{center}
  $ \frac{1}{m_{T}}\frac{dN}{dm_{T}}\sim \exp^{-\beta m_{T}}$
\end{center}
vhodné pre popis experimentálnych dát  na malom intervale rapidít, zatiaľ čo
rozdelenie
\begin{center}
  $ \frac{1}{m_{T}^{3/2}}\frac{dN}{dm_{T}}\sim \exp^{-\beta m_{T}} $
\end{center}
by  malo vyhovovať pre rapidity v rozmedzí niekoľkých jednotiek, t.j.  malo
by byť obecnejšie.

Rozdelenia priečnych hmotností $K^{0}_{S}$ mezónov boli získané v oblasti
rapidít $2.5<y<3.3$ a priečnych hybností $1<p_{T}<3\: GeV/c$. Výsledné
nekorigované a korigované rozdelenia v logaritmickej škále ukazuje obr. 6.5.

\newpage
\begin{center}
  \includegraphics*[bb=2 3 560 560,width=7cm]{mt.eps}
  \includegraphics*[bb=2 3 560 560,width=7cm]{mtc.eps}
\end{center}
\begin{center}
  Obrázok 6.5: $m_{T}$ spektrum $K^{0}_{S}$ mezónov.
\end{center}
Pri porovnaní nekorigovaných, obr. 6.5 a) a korigovaných, obr. 6.5 b)
$m_{T}$  spektier vidno, že korekcia priaznivo ovplyvnila linearitu
rozdelenia (v logaritmickej škále).

Experimentálne údaje sa fitujú funkciou
\begin{equation}
  \frac{dN}{dm_{T}} \sim m_{T}^{\alpha}\exp^{-\beta m_{T}},
\end{equation}
kde pre $\alpha$ sa používajú tri hodnoty $\alpha=1/2, 1, 3/2$.
% Ak takéto korigované spektrum $m_{T}$ fitujeme funkciou 6.1, dostávame
% hodnotu teploty 208 $\pm$ 3.3 MeV. Pri fitovaní  funkciou 6.2 hodnoty
% teploty sú 197 $\pm$ 2.8 MeV .
Výsledky fitovania sú zobrazené v tab. 6.1.

%\vspace*{0.7cm}
\begin{center}
  \begin{tabular}{|c|c|c|}
    \hline
    $\alpha$ & teplota [MeV] & $\chi^{2}$ \\
    \hline
    1/2 & 220 $\pm$ 3.4 & 1.374 \\ \hline
    1 & 208 $\pm$ 3.3 & 1.461 \\ \hline
    3/2 & 197 $\pm$ 2.8 & 1.591 \\ \hline

  \end{tabular}
\end{center}
\begin{center}
  Tabuľka 6.1: Výsledná teplota v experimente WA94.
\end{center}



Okrajové body boli pri fitovaní vylúčené.  Chyby, ktoré sú uvedené, sú len
štatistické chyby. Analýza produkcie   $K^{0}_{S}$ mezónov v experimente WA94
ešte nebola robená a výsledky neboli publikované. Preto porovnanie môžeme
urobiť pre iný druh častice alebo pre iný experiment. Tieto porovnania
sumarizujú tabuľky 6.2 a 6.3 [20,21,22,23].

\newpage
%\vspace*{0.7cm}
\begin{center}
  \shorthandoff{-}   % fix problem with cline + slovak babel
  \begin{tabular}{|l|c|c|}
    \hline
    častica & teplota[MeV] & rapidita  \\
    \hline \hline
    $\Lambda$ & 213 $\pm$ 3 & 2.5$<y<$3.0  \\ \cline{2-3}
    $\overline{\Lambda}$ & 204 $\pm$ 5  & 2.5$<y<$3.0 \\ \cline{2-3}
    $K^{+}$ & 172 $\pm$ 20 & 2.7$<y<$3.2 \\ \cline{2-3}
    $K^{-}$ & 152 $\pm$ 25 & 2.7$<y<$3.2 \\ \cline{2-3}
    $K^{0}_{S}$ & {\bf 197  ${\pm}$   3} & 2.5$<y<$3.3 \\
    \hline

  \end{tabular}
\end{center}
\begin{center}
  Tabuľka 6.2: Teploty získané z experimentu WA94, $\alpha=3/2$.
\end{center}

\begin{center}
  \begin{tabular}{|l|c|c|}
    \hline
    experiment & teplota [MeV] & rapidita \\ \hline \hline
    NA34 (pS) & 206 & 1.7$<y<$4.2 \\
    NA34 (SS) & 193 & 1.4$<y<$2.7 \\
    NA35 (pS) & 205 $\pm$ 16 & 0.0$<y<$5.0 \\
    NA35 (SS) & 193 & 1.0$<y<$2.5 \\
    NA35 (pAu) & 210 $\pm$ 15 & 2.2$<y<$5.0 \\
    NA35 (OAu) & 238 $\pm$ 23 & 1.5$<y<$2.5 \\
    WA85 (pW) & 224 $\pm$ 3 & 2.3$<y<$3.8 \\
    WA85 (SW) & 219 $\pm$ 5 & 2.3$<y<$3.8 \\
    WA94 (SS) & {\bf 197 $\pm$ 3} & 2.5$<y<$3.3 \\
    WA97 (pPb) & 197 $\pm$ 5 & 2.9$<y<$3.6 \\
    WA97 (PbPb) & 210 $\pm$ 7 & 2.9$<y<$3.6 \\
    \hline
  \end{tabular}
\end{center}

\begin{center}
  Tabuľka 6.3: Teploty $K^{0}$ pre rôzne experimenty, $\alpha=3/2$.
\end{center}

Experimentálne výsledky pre sklon priečnej hmotnosti získané v tejto práci
sú konzistentné s výsledkami  experimentov NA34, NA35, v ktorých sa skúmali
zrážky S-S pri energiách 200 GeV/c na nukleón.

Pri porovnávaní výsledkov interakcie rôznych jadier (viď tab. 6.3)  však
musíme byť opatrní. Veľkosť  prevrátenej hodnoty sklonu (\uv{teploty}) je
závislá od oblasti rapidít a priečnych hybností. Experimentálne údaje
uvedené v tabuľke nie sú získané v rovnakej oblasti $y$ a $p_{T}$.
Hodnoty $p_{T}$ pre tieto experimenty sa zhruba nachádzajú v intervale
1$<p_{T}<$3 GeV/c. Odhliadnuc od tohto faktu je tu určitý náznak toho, že
\uv{teplota} $K^{0}$ môže rásť s veľkosťou zrážajúcih sa objektov. Výnimku
tvoria údaje z experimentu WA97 (Pb-Pb pri 160 GeV/c na nukleón), kde
rapiditný interval je najmenší. V rámci troch experimentálnych chýb môžeme
ovšem  \uv{teplotu} považovať za rovnakú, nezávislú od rozmerov jadier.
% Napriek tomu tu môžeme vidieť určitú závislosť. Teplota
% v oblasti, kde sa formovali častice, rastie s rozmerom interagujúcich iónov.
% Tiež je tu vidieť, že pri fitovaní, kde parameter $\alpha$ je rovný 1, je
% značný nárast teploty   oproti fitu s parametrom $\alpha=3/2$.
%

\chapter{Záver}
Práca sa zaoberala štúdiom produkcie $K^{0}_{S}$ mezónov v zrážkach jadier
S-S pri hybnostiach 200 GeV/c na nukleón. Získanie  údajov prebiehalo v
rámci experimentu WA94 v CERNe.
% Analýza sa
% vykonávala v oblasti priečných hybnosí $1<p_{T}<3\: GeV/c$ a rapidít
% $2.5<y<3.3$.  Boli získane experimentálne rozdelenia priečných hmotností
% $m_{T}$, efektívnych hmotnosti $m$, priečných hybností $p_{T}$ a  rapidít
% $y$. Experimentálne údaje boli korigované na vypočítanú akceptanciu a
% efektívnosť rekonštrukcie. Ukázalo sa, že zvolené kritéria rozumne vyberajú
% $K^{0}$ kandidátov. Rozdelenia $m_{T}$ boli fitované dvoma závislosťami a
% boli porovnané s iným experimentom. Z fitu tohto rozdelenia sme získali
% teplotu excitovanej jadrovej hmoty, 197 $\pm$ 3 MeV.  Taktiež sa skúmal vplyv
% orientácie magnetického poľa Omega na analýzu údajov. Ukázalo sa, že túto
% orientáciu pri analýze nie je potrebné diferencovať.
%
V diplomovej práci boli získané nasledujúce výsledky:
\begin{itemize}
\item[-] výpočet akceptancie $K^{0}_{S}$ mezónov,
\item[-] výpočet efektívnosti rekonštrukcie,
\item[-] porovnanie výsledkov pre magnetické pole hore a dole, výsledky
  pre obidve orientácie poľa sú kompatibilné,

  %  \item[-] experimentálne údaje korigované na akceptanciu a efektívnosť
  %   rekonštrukcie
\item[-] rozdelenia invariantných hmotností, priečnych hybností $p_{T}$,
  rapidít $y$ a prie\-čnych hmotností $m_{T}$;  rozumná identifikácia
  $K^{0}_{S}$ mezónov na základe použitých kritérií,
\item[-] rozdelenie $m_{T}$ bolo fitované funkciou 6.1 a boli získané
  rôzne hodnoty \uv{teploty},
\item[-] porovnanie s výsledkami iných experimentov - sklon rozdelenia
  priečnych hmotností $K^{0}_{S}$ získaných v tejto práci je v dobrom súlade
  s výsledkami iných experimentov skúmajúcich tú istú zrážku. Na základe
  experimentálnych výsledkov z interakcií rôznych jadier nie je môžné urobiť
  jednoznačný záver o závislosti \uv{teploty} od interagujúcich jadier.
\end{itemize}


%\begin{thebibliography}{99}
\bibitem{} Halzen F., Martin A.D.: {\em Quarks and Leptons: An Introductory
  of Course in Modern Particle Physics}, John Wiley and Sons, 1984.
\bibitem{} Rafelski J.: {\em 21st Rencontres de Moriond}, Les Arcs, 1983
\bibitem{} Jacob M.: {\em In Search of Quark Gluon Plasma}, Springer Verlag,
  1985
\bibitem{} Davies J.P.: {\em Particles in Ultra-Relativistic Proton-Tungsten
  Collision at 200 GeV/c per Nucleon}, PhD Thesis, School of Physics and Space
  Research, Faculty of Sciense, The University of Birmingham, November 1995

\bibitem{} Bayes, C.A.: {\em Strange Particle Production in
  Sulphur-Sulphur Interactions at 200
  GeV/c per Nucleon}, PhD Thesis, RAL-TH-95-008, School of Physics and Space
  Research, Faculty of Sciense, The University of Birmingham, March 1995.
\bibitem{} Barnes, P. R.: {\em Production of Strange and Multistarnge
  Baryons in Ultrarelativistic Heavy Ion Collision}, PhD Thesis, RALT-169,
  Scool of Physics and Space Research, Faculty of Science, The University of
  Birmingham, June 1993.
\bibitem{} Matsui T.: {\em Phys. Lett. {\bf B178} 1986 416.}
\bibitem{} Bordalo P.: {\em NA38 Collaboration}, Proc. XXII Rencontres de
  Moriond, 1988.
\bibitem{} Rafelski J. and Hagedorn R.: {\em Statistical Mechanics of Quarks
  and  Hadrons}, North Holland, Amsterdam 1981.

\bibitem{} Evans, D.: {\em  Strange and Multistarnge Baryon and Antibaryon
  Production in Sulphur-Tungsten Interactions at 200 GeV/c per Nucleon}, PhD
  Thesis, RALT-130, School of Physics and Space Research, The University of
  Birmingham, April 1992.
\bibitem{} Ellis J. and Heinz U.: {\em Phys. Lett. {\bf B233} 1989 167.}
\bibitem{} Eggers H.C. and Rafelski J.: {\em Int. J. of Mod. Phys. A,
  Vol.6, No. 7 1991 1067.}
\bibitem{} Koch P., Müller B. and Rafelski J.: {\em Phys. Rep. 142 1986
  167.}
\bibitem{} Hove van L.: {\em Z. Phys. {\bf C27} 1985 135.}
\bibitem{} Csernai L.P., Holme A.K. and Staubo E.F.: {\em What can we learn
  from $p_{T}$ distribution in relativistic heavy ion collisions ?}, Plenum
  Press, 1989.
\bibitem{} The WA85 Coll., Abatzis et al.: {\em Phys. Lett. {\bf B244} 1990
  130.}
\bibitem{} Lassalle J. C. at al.: {\em DD/EE/79-2 (TRIDENT manual)}
\bibitem{} Eggers H. C. and Rafelski J.: {\em Int. J. of Mod. Phys. A, vol.
  6, no. 7 (1991) 1067}
\bibitem{} Schukraft J.: {\em CERN-PPE/91-04}
\bibitem{} Kinson J. B.: {\em Nuclear Physics {\bf A590} (1991) 317c.-332c.}
\bibitem{} Lietava R.: {\em J. Phys. G: Nucl. Part. Phys.{\bf 25}
  (1999) 181-188},
\bibitem{} HELIOS Collaboration {\em Phys. Lett. {\bf B 296} (1992) 277}
\bibitem{} NA35 Collaboration {\em Z. Phys. {\bf C 64} (1994) 197-200}


\bibitem{} Helstrup, H.: {\em Starnge and Non-Strange Baryon Production
  in Ultrarelativistic Sulphur-Tungsten and Sulphur-Sulphur Collision}, PhD
  Thesis, Fysisk institutt, Universitetet i Bergen, April 1993.

\bibitem{} Lovhoiden, G.: {\em Heavy Ion Collision at High Energies}, in
  Kompendium FYS338 (UiB) og FYS374 (UiO), 1996.
\end{thebibliography}

%\end{document}
