Na obr. 4.2a je znázornený rozpad $V^{0}$ v~laboratórnej sústave na častice
s~kladným (+) a záporným (-)  nábojom s~pozdĺžnou hybnosťou  $p_{L}$ v~smere
letu $V^{0}$ a priečnou hybnosťou $p_{T}$. V~CMS sústave, obr. 4.2b,  zložky
hybností kladnej (+) častice majú tvar
\begin{center}
  $ p_{L}^{*}(+)=p^{*}\cos\theta^{*}$ \\
  $ p_{T}^{*}(+)=p^{*}\sin\theta^{*}$
\end{center}
a energia
\begin{center}
  $ E^{*}(+)=\sqrt{p^{{*}^{2}} + m^{2}(+)}$.
\end{center}

\newpage
\begin{center}
  Na obr. 4.2a je znázornený rozpad $V^{0}$ v~laboratórnej sústave na častice 
s~kladným (+) a záporným (-)  nábojom s~pozdĺžnou hybnosťou  $p_{L}$ v~smere
letu $V^{0}$ a priečnou hybnosťou $p_{T}$. V~CMS sústave, obr. 4.2b,  zložky
hybností kladnej (+) častice majú tvar
\begin{center}
  $ p_{L}^{*}(+)=p^{*}\cos\theta^{*}$ \\  
  $ p_{T}^{*}(+)=p^{*}\sin\theta^{*}$
\end{center}
a energia 
\begin{center}
  $ E^{*}(+)=\sqrt{p^{{*}^{2}} + m^{2}(+)}$.
\end{center}

\newpage
\begin{center}
Na obr. 4.2a je znázornený rozpad $V^{0}$ v~laboratórnej sústave na častice 
s~kladným (+) a záporným (-)  nábojom s~pozdĺžnou hybnosťou  $p_{L}$ v~smere
letu $V^{0}$ a priečnou hybnosťou $p_{T}$. V~CMS sústave, obr. 4.2b,  zložky
hybností kladnej (+) častice majú tvar
\begin{center}
  $ p_{L}^{*}(+)=p^{*}\cos\theta^{*}$ \\  
  $ p_{T}^{*}(+)=p^{*}\sin\theta^{*}$
\end{center}
a energia 
\begin{center}
  $ E^{*}(+)=\sqrt{p^{{*}^{2}} + m^{2}(+)}$.
\end{center}

\newpage
\begin{center}
Na obr. 4.2a je znázornený rozpad $V^{0}$ v~laboratórnej sústave na častice 
s~kladným (+) a záporným (-)  nábojom s~pozdĺžnou hybnosťou  $p_{L}$ v~smere
letu $V^{0}$ a priečnou hybnosťou $p_{T}$. V~CMS sústave, obr. 4.2b,  zložky
hybností kladnej (+) častice majú tvar
\begin{center}
  $ p_{L}^{*}(+)=p^{*}\cos\theta^{*}$ \\  
  $ p_{T}^{*}(+)=p^{*}\sin\theta^{*}$
\end{center}
a energia 
\begin{center}
  $ E^{*}(+)=\sqrt{p^{{*}^{2}} + m^{2}(+)}$.
\end{center}

\newpage
\begin{center}
\input{alfa.pic}
\end{center}
\vspace{-6cm}
\begin{center}
  Obrázok 4.2: Schématické znázornenie $V^{0}$ rozpadu.
\end{center}
Transformácia do laboratórnej sústavy pomocou Lorentzovych transformácií má
v~maticovom zápise  tvar
\begin{center}
  $$ \Biggl( \begin{array}{c}
     E \\
     p_{L} \end{array} \Biggr) = \Biggl( \begin{array}{cc} \gamma &
\gamma\beta \\ \gamma\beta & \gamma \end{array} \Biggr) \Biggl(
\begin{array}{c} E^{*} \\ p_{L}^{*} \end{array} \Biggr), \hspace{1cm}
p_{T}^{*}= p_{T} $$
\end{center}
kde $\gamma=1/\sqrt{1-\beta^{2}}$. Transformáciou dostávame 
\begin{center}
  $E^{Lab}(+)=\gamma E^{*}(+) + \gamma\beta p_{L}^{*}(+)$ \\
  $p_{L}^{Lab}(+)=\gamma p_{L}^{*}(+) + \gamma\beta E^{*}(+) = \gamma
p^{*}\cos
  \theta^{*} + \gamma\beta E^{*}(+)$ \\
  $p_{T}^{Lab}(+)=p_{T}^{*}(+)= p^{*}\sin\theta^{*},$
\end{center}
to znamená, že pozdĺžná hybnosť pre kladnú (+) a zápornú (-) časticu má tvar
\begin{center}
  $p_{L}^{Lab}(+)= \gamma p^{*}\cos  \theta^{*} + \gamma\beta E^{*}(+)$ \\
  $p_{L}^{Lab}(-)= -\gamma p^{*}\cos  \theta^{*} + \gamma\beta E^{*}(-)$
\end{center}
potom dostávame 
\begin{center}
  $p_{L}^{Lab}(+)-p_{L}^{Lab}(-)=2\gamma p^{*}\cos\theta^{*} +\beta\gamma
(E^{*}(+)-E^{*}(-)) $ \\
  $p_{L}^{Lab}(+)+p_{L}^{Lab}(+)= p_{L}^{Lab}=\beta\gamma m_{{V}^{0}}$.
\end{center}
Armenteros--Podolanského $\alpha$ je definovaná ako
\begin{center}
 $\alpha={\Large\frac{p_{L}^{Lab}(+)-p_{L}^{Lab}(-)}{p_{L}^{Lab}(+)+p_{L}^{Lab}(+)}}
= {\Large\frac{2p^{*}\cos\theta^{*}}{\beta m_{{V}^{0}}}}+{\Large\frac{E^{*}(+)-E^{*}(-)}
{m_{{V}^{0}}}} = \zeta\cos\theta^{*} + \varphi$
\end{center}
to znamená 
\begin{center}
  $\cos\theta^{*}={\Large\frac{\alpha-\varphi}{\zeta}}$ \hspace{1cm} $\sin\theta^{*}
= {\Large\frac{p_{T}^{*}}{p^{*}}}$ \\
  $\cos^{2}\theta^{*} + \sin^{2}\theta^{*} =1={\Large \bigl(\frac{\alpha-\varphi}
{\zeta} \bigr)^{2}} + {\Large\bigl(\frac{p_{T}^{*}}{p^{*}} \bigr)^{2}}$.
\end{center}
Tento posledný výraz je rovnica elipsy so stredom v bode $(\varphi,0)$ a
poloosmi v~smere $\alpha$ veľkosti $\zeta$ a $p_{T}$ veľkosti $p^{*}$. V tab.
4.1. sú tieto hodnoty pre rôzne $V^{0}$ a na obr. 4.3  sú znázornené elipsy
pre rozpady $\Lambda$, $\overline{\Lambda}$ a $K^{0}$.

\vspace*{0.5cm}
\begin{tabular}{|l|c|c|c|c|c|}
\hline
Rozpad & $\zeta$ & $p^{*}[GeV/c]$ & $\alpha_{min}$ & $\alpha_{max}$ & 
$\varphi$ \\ \hline
$K^{0}_{S}\rightarrow\pi^{+}\pi^{-}$ & 0.8282 & 0.206 & -0.8282 & +0.8282 &
0 \\
$\Lambda\rightarrow p\pi^{-}$ & 0.179 & 0.101 & +0.515 & +0.873 & 0.694 \\
$\overline{\Lambda}\rightarrow \overline{p}\pi^{+}$ & 0.179 & 0.101 &
-0.873 & -0.515 & -0.694 \\ \hline  
\end{tabular}
\begin{center}
  Tabuľka 4.1: Armenterovské premenné.
\end{center}

\end{center}
\vspace{-6cm}
\begin{center}
  Obrázok 4.2: Schématické znázornenie $V^{0}$ rozpadu.
\end{center}
Transformácia do laboratórnej sústavy pomocou Lorentzovych transformácií má
v~maticovom zápise  tvar
\begin{center}
  $$ \Biggl( \begin{array}{c}
     E \\
     p_{L} \end{array} \Biggr) = \Biggl( \begin{array}{cc} \gamma &
\gamma\beta \\ \gamma\beta & \gamma \end{array} \Biggr) \Biggl(
\begin{array}{c} E^{*} \\ p_{L}^{*} \end{array} \Biggr), \hspace{1cm}
p_{T}^{*}= p_{T} $$
\end{center}
kde $\gamma=1/\sqrt{1-\beta^{2}}$. Transformáciou dostávame 
\begin{center}
  $E^{Lab}(+)=\gamma E^{*}(+) + \gamma\beta p_{L}^{*}(+)$ \\
  $p_{L}^{Lab}(+)=\gamma p_{L}^{*}(+) + \gamma\beta E^{*}(+) = \gamma
p^{*}\cos
  \theta^{*} + \gamma\beta E^{*}(+)$ \\
  $p_{T}^{Lab}(+)=p_{T}^{*}(+)= p^{*}\sin\theta^{*},$
\end{center}
to znamená, že pozdĺžná hybnosť pre kladnú (+) a zápornú (-) časticu má tvar
\begin{center}
  $p_{L}^{Lab}(+)= \gamma p^{*}\cos  \theta^{*} + \gamma\beta E^{*}(+)$ \\
  $p_{L}^{Lab}(-)= -\gamma p^{*}\cos  \theta^{*} + \gamma\beta E^{*}(-)$
\end{center}
potom dostávame 
\begin{center}
  $p_{L}^{Lab}(+)-p_{L}^{Lab}(-)=2\gamma p^{*}\cos\theta^{*} +\beta\gamma
(E^{*}(+)-E^{*}(-)) $ \\
  $p_{L}^{Lab}(+)+p_{L}^{Lab}(+)= p_{L}^{Lab}=\beta\gamma m_{{V}^{0}}$.
\end{center}
Armenteros--Podolanského $\alpha$ je definovaná ako
\begin{center}
 $\alpha={\Large\frac{p_{L}^{Lab}(+)-p_{L}^{Lab}(-)}{p_{L}^{Lab}(+)+p_{L}^{Lab}(+)}}
= {\Large\frac{2p^{*}\cos\theta^{*}}{\beta m_{{V}^{0}}}}+{\Large\frac{E^{*}(+)-E^{*}(-)}
{m_{{V}^{0}}}} = \zeta\cos\theta^{*} + \varphi$
\end{center}
to znamená 
\begin{center}
  $\cos\theta^{*}={\Large\frac{\alpha-\varphi}{\zeta}}$ \hspace{1cm} $\sin\theta^{*}
= {\Large\frac{p_{T}^{*}}{p^{*}}}$ \\
  $\cos^{2}\theta^{*} + \sin^{2}\theta^{*} =1={\Large \bigl(\frac{\alpha-\varphi}
{\zeta} \bigr)^{2}} + {\Large\bigl(\frac{p_{T}^{*}}{p^{*}} \bigr)^{2}}$.
\end{center}
Tento posledný výraz je rovnica elipsy so stredom v bode $(\varphi,0)$ a
poloosmi v~smere $\alpha$ veľkosti $\zeta$ a $p_{T}$ veľkosti $p^{*}$. V tab.
4.1. sú tieto hodnoty pre rôzne $V^{0}$ a na obr. 4.3  sú znázornené elipsy
pre rozpady $\Lambda$, $\overline{\Lambda}$ a $K^{0}$.

\vspace*{0.5cm}
\begin{tabular}{|l|c|c|c|c|c|}
\hline
Rozpad & $\zeta$ & $p^{*}[GeV/c]$ & $\alpha_{min}$ & $\alpha_{max}$ & 
$\varphi$ \\ \hline
$K^{0}_{S}\rightarrow\pi^{+}\pi^{-}$ & 0.8282 & 0.206 & -0.8282 & +0.8282 &
0 \\
$\Lambda\rightarrow p\pi^{-}$ & 0.179 & 0.101 & +0.515 & +0.873 & 0.694 \\
$\overline{\Lambda}\rightarrow \overline{p}\pi^{+}$ & 0.179 & 0.101 &
-0.873 & -0.515 & -0.694 \\ \hline  
\end{tabular}
\begin{center}
  Tabuľka 4.1: Armenterovské premenné.
\end{center}

\end{center}
\vspace{-6cm}
\begin{center}
  Obrázok 4.2: Schématické znázornenie $V^{0}$ rozpadu.
\end{center}
Transformácia do laboratórnej sústavy pomocou Lorentzovych transformácií má
v~maticovom zápise  tvar
\begin{center}
  $$ \Biggl( \begin{array}{c}
     E \\
     p_{L} \end{array} \Biggr) = \Biggl( \begin{array}{cc} \gamma &
\gamma\beta \\ \gamma\beta & \gamma \end{array} \Biggr) \Biggl(
\begin{array}{c} E^{*} \\ p_{L}^{*} \end{array} \Biggr), \hspace{1cm}
p_{T}^{*}= p_{T} $$
\end{center}
kde $\gamma=1/\sqrt{1-\beta^{2}}$. Transformáciou dostávame 
\begin{center}
  $E^{Lab}(+)=\gamma E^{*}(+) + \gamma\beta p_{L}^{*}(+)$ \\
  $p_{L}^{Lab}(+)=\gamma p_{L}^{*}(+) + \gamma\beta E^{*}(+) = \gamma
p^{*}\cos
  \theta^{*} + \gamma\beta E^{*}(+)$ \\
  $p_{T}^{Lab}(+)=p_{T}^{*}(+)= p^{*}\sin\theta^{*},$
\end{center}
to znamená, že pozdĺžná hybnosť pre kladnú (+) a zápornú (-) časticu má tvar
\begin{center}
  $p_{L}^{Lab}(+)= \gamma p^{*}\cos  \theta^{*} + \gamma\beta E^{*}(+)$ \\
  $p_{L}^{Lab}(-)= -\gamma p^{*}\cos  \theta^{*} + \gamma\beta E^{*}(-)$
\end{center}
potom dostávame 
\begin{center}
  $p_{L}^{Lab}(+)-p_{L}^{Lab}(-)=2\gamma p^{*}\cos\theta^{*} +\beta\gamma
(E^{*}(+)-E^{*}(-)) $ \\
  $p_{L}^{Lab}(+)+p_{L}^{Lab}(+)= p_{L}^{Lab}=\beta\gamma m_{{V}^{0}}$.
\end{center}
Armenteros--Podolanského $\alpha$ je definovaná ako
\begin{center}
 $\alpha={\Large\frac{p_{L}^{Lab}(+)-p_{L}^{Lab}(-)}{p_{L}^{Lab}(+)+p_{L}^{Lab}(+)}}
= {\Large\frac{2p^{*}\cos\theta^{*}}{\beta m_{{V}^{0}}}}+{\Large\frac{E^{*}(+)-E^{*}(-)}
{m_{{V}^{0}}}} = \zeta\cos\theta^{*} + \varphi$
\end{center}
to znamená 
\begin{center}
  $\cos\theta^{*}={\Large\frac{\alpha-\varphi}{\zeta}}$ \hspace{1cm} $\sin\theta^{*}
= {\Large\frac{p_{T}^{*}}{p^{*}}}$ \\
  $\cos^{2}\theta^{*} + \sin^{2}\theta^{*} =1={\Large \bigl(\frac{\alpha-\varphi}
{\zeta} \bigr)^{2}} + {\Large\bigl(\frac{p_{T}^{*}}{p^{*}} \bigr)^{2}}$.
\end{center}
Tento posledný výraz je rovnica elipsy so stredom v bode $(\varphi,0)$ a
poloosmi v~smere $\alpha$ veľkosti $\zeta$ a $p_{T}$ veľkosti $p^{*}$. V tab.
4.1. sú tieto hodnoty pre rôzne $V^{0}$ a na obr. 4.3  sú znázornené elipsy
pre rozpady $\Lambda$, $\overline{\Lambda}$ a $K^{0}$.

\vspace*{0.5cm}
\begin{tabular}{|l|c|c|c|c|c|}
\hline
Rozpad & $\zeta$ & $p^{*}[GeV/c]$ & $\alpha_{min}$ & $\alpha_{max}$ & 
$\varphi$ \\ \hline
$K^{0}_{S}\rightarrow\pi^{+}\pi^{-}$ & 0.8282 & 0.206 & -0.8282 & +0.8282 &
0 \\
$\Lambda\rightarrow p\pi^{-}$ & 0.179 & 0.101 & +0.515 & +0.873 & 0.694 \\
$\overline{\Lambda}\rightarrow \overline{p}\pi^{+}$ & 0.179 & 0.101 &
-0.873 & -0.515 & -0.694 \\ \hline  
\end{tabular}
\begin{center}
  Tabuľka 4.1: Armenterovské premenné.
\end{center}

\end{center}
\vspace{-6cm}
\begin{center}
  Obrázok 4.2: Schématické znázornenie $V^{0}$ rozpadu.
\end{center}
Transformácia do laboratórnej sústavy pomocou Lorentzovych transformácií má
v~maticovom zápise  tvar
\begin{center}
  $$ \Biggl( \begin{array}{c}
    E \\
    p_{L} \end{array} \Biggr) = \Biggl( \begin{array}{cc} \gamma &
    \gamma\beta \\ \gamma\beta & \gamma \end{array} \Biggr) \Biggl(
  \begin{array}{c} E^{*} \\ p_{L}^{*} \end{array} \Biggr), \hspace{1cm}
  p_{T}^{*}= p_{T} $$
\end{center}
kde $\gamma=1/\sqrt{1-\beta^{2}}$. Transformáciou dostávame
\begin{center}
  $E^{Lab}(+)=\gamma E^{*}(+) + \gamma\beta p_{L}^{*}(+)$ \\
  $p_{L}^{Lab}(+)=\gamma p_{L}^{*}(+) + \gamma\beta E^{*}(+) = \gamma
  p^{*}\cos
  \theta^{*} + \gamma\beta E^{*}(+)$ \\
  $p_{T}^{Lab}(+)=p_{T}^{*}(+)= p^{*}\sin\theta^{*},$
\end{center}
to znamená, že pozdĺžná hybnosť pre kladnú (+) a zápornú (-) časticu má tvar
\begin{center}
  $p_{L}^{Lab}(+)= \gamma p^{*}\cos  \theta^{*} + \gamma\beta E^{*}(+)$ \\
  $p_{L}^{Lab}(-)= -\gamma p^{*}\cos  \theta^{*} + \gamma\beta E^{*}(-)$
\end{center}
potom dostávame
\begin{center}
  $p_{L}^{Lab}(+)-p_{L}^{Lab}(-)=2\gamma p^{*}\cos\theta^{*} +\beta\gamma
  (E^{*}(+)-E^{*}(-)) $ \\
  $p_{L}^{Lab}(+)+p_{L}^{Lab}(+)= p_{L}^{Lab}=\beta\gamma m_{{V}^{0}}$.
\end{center}
Armenteros--Podolanského $\alpha$ je definovaná ako
\begin{center}
  $\alpha={\Large\frac{p_{L}^{Lab}(+)-p_{L}^{Lab}(-)}{p_{L}^{Lab}(+)+p_{L}^{Lab}(+)}}
  = {\Large\frac{2p^{*}\cos\theta^{*}}{\beta m_{{V}^{0}}}}+{\Large\frac{E^{*}(+)-E^{*}(-)}
    {m_{{V}^{0}}}} = \zeta\cos\theta^{*} + \varphi$
\end{center}
to znamená
\begin{center}
  $\cos\theta^{*}={\Large\frac{\alpha-\varphi}{\zeta}}$ \hspace{1cm} $\sin\theta^{*}
  = {\Large\frac{p_{T}^{*}}{p^{*}}}$ \\
  $\cos^{2}\theta^{*} + \sin^{2}\theta^{*} =1={\Large \bigl(\frac{\alpha-\varphi}
    {\zeta} \bigr)^{2}} + {\Large\bigl(\frac{p_{T}^{*}}{p^{*}} \bigr)^{2}}$.
\end{center}
Tento posledný výraz je rovnica elipsy so stredom v bode $(\varphi,0)$ a
poloosmi v~smere $\alpha$ veľkosti $\zeta$ a $p_{T}$ veľkosti $p^{*}$. V tab.
4.1. sú tieto hodnoty pre rôzne $V^{0}$ a na obr. 4.3  sú znázornené elipsy
pre rozpady $\Lambda$, $\overline{\Lambda}$ a $K^{0}$.

\vspace*{0.5cm}
\begin{tabular}{|l|c|c|c|c|c|}
  \hline
  Rozpad & $\zeta$ & $p^{*}[GeV/c]$ & $\alpha_{min}$ & $\alpha_{max}$ &
  $\varphi$ \\ \hline
  $K^{0}_{S}\rightarrow\pi^{+}\pi^{-}$ & 0.8282 & 0.206 & -0.8282 & +0.8282 &
  0 \\
  $\Lambda\rightarrow p\pi^{-}$ & 0.179 & 0.101 & +0.515 & +0.873 & 0.694 \\
  $\overline{\Lambda}\rightarrow \overline{p}\pi^{+}$ & 0.179 & 0.101 &
  -0.873 & -0.515 & -0.694 \\ \hline
\end{tabular}
\begin{center}
  Tabuľka 4.1: Armenterovské premenné.
\end{center}
