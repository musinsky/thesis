\chapter{QGP a jej príznaky}
Jednou z~najdôležitejších predpoved\'\i\ QCD je možnosť
prechodu hadrónovej hmo\-ty pri extrémnych teplotách a hustotách
do stavu kvark-gluónovej plazmy, do stavu asymptoticky voľného
plynu kvarkov a gluónov. Štúdium plazmy by preto mohlo viac 
objasni\softt{} také javy QCD ako väznenie a uvoľnenie
kvarkov  a gluónov v~extrémnych podmienkach. Existencia
plazmy by preto mohla by\softt{} dobrým testom kvantovej chromodynamiky.
%\vspace*{11.4cm}

\begin{center}
\includegraphics*[bb=60 345 495 650,width=14cm]{fazdiag.eps}
\end{center}

\begin{center}
Obrázok 2.1: Fázový diagram ukazujúci prechod hadrónovej hmoty na QGP.
\end{center}
%obrazok z Barnesa z 7 str.
\newpage
Ako vidieť z~obr. 2.1 je tu viac možnosti formovania plazmy [2]:

\begin{itemize}
\item[-] stlačením studenej jadrovej hmoty, t.j. zvyšovaním
hustoty hadrónov, ako sa predpokladá  v~prípade  jadier neutrónových
hviezd,
\item[-] ohrievaním nukleárnej hmoty normálnej hustoty,  čo
môže by\softt{} opakom historickej  evolúcie vo vesmíre,
\item[-] alebo akýmsi kompromisom medzi týmito dvoma extrémnymi 
prípadmi, t.j. kombináciou ohrievania a kompresie, čo sa
predpokladá pri zrážkach ťažkých iónov dostatočne vysokých
energií.
\end{itemize}

Výskum QGP má význam nielen z~hľadiska fyziky vysokých energii, 
ale aj napr. v~astrofyzike, kde sa predpokladá, že stred neutrónovej 
hviezdy môže  by\softt{} dostatočne hustý na to, aby tam existovala
plazma. Pravdepodobne až do 10$^{-5}$~s po Big Bangu, ke\softd{} bola
teplota okolo 200 MeV (čo je predpokladaná minimálna teplota 
potrebná na formovanie plazmy), vesmír mohol existova\softt{} ako QGP
[3]. Ak  skutočne bol vesmír vo svojich ranných
štádiach QGP, potom správanie sa  plazmy, ako sa začal vesmír
ochladzova\softt{} a hadronizova\softt{}, by mohlo vies\softt{}  k~pochopeniu
nehomogénnosti dnešného vesmíru. 

Za účelom výskumu plazmy boli navrhnuté vysokoenergetické 
zrážky \softt{}ažkých  iónov, kde pri ve\softl{}kom
množstve interakcií nukleónov je väčšia pravdepodobnosť, 
že vzniknutý  fireball, z ktorého môže QGP vznikať, bude ma\softt{}
dostatočnú teplotu a hustotu [3]. 


Urýchľovanie \softt{}ažkých iónov sa stalo možným v~CERNe a 
Brookhavene v~1986.  Urých\softl{}ované boli ióny, kde pomer
protonového a nukleonového čísla bol Z/A=1/2. V~Brookhavene to
boli kyslíkové, zlaté a kremíkové zväzky urýchlené na 14.5
Gev/c na nukleón, zatia\softl{}čo v~CERNe kyslíkové a sírové
ióny  do energií 200 GeV/c na nukleón a neskoršie
aj ióny olova do energií 160 GeV/c na nukleón.
% \section{Dynamika zrážok \softt{}ažkých iónov.}
% Na obrázku 2.2 [4] môžeme vidie\softt{} schematické znázornenie 
% zrážky dvoch \softt{}ažkých iónových jadier. Pre štúdium
% týchto zrážok je užitočné zavies\softt{} premennú 
% rapidita, definovanú ako 
% \begin{center}{\large $Y=\frac{1}{2}\ln(\frac{E^{*}+p_{L}^{*}}
% {E^{*}-p_{L}^{*}})$}
% \end{center}
% kde $E^{*}$ je energia v~CMS a $p_{L}^{*}$ je pozdlžná hybnos\softt{} 
% v~CMS. Táto  premenná je invariantná voči Lorentzovým
% transformáciam. Pre symetrický  kolizujúci systém (tu patria
% zrážky síra-síra WA94) je $p_{L}$=0, čomu  zodpovedá
% $Y^{*}$=0 v~CMS a $Y_{lab}\sim$2.8 v~laboratórnej sústave. 
% 
% Informácie o~procesoch v~zrážkach nám poskytne detekcia 
% častíc. Vhodné je používať priečne hodnoty
% premenných akými sú: priečna hybnosť $p_{T}$, priečna
% energia $E_{T}$ a priečna hmotnosť definovaná ako $m_{T}=
% \sqrt{p_{T}^{2} + m^{2}}$. Teplota systému je funkciou priečnej
% hmotnosti. 
% 
% V~hlboko nepružnej zrážke ve\softl{}ké množstvo energie a 
% hmoty môže by\softt{}  pretransformované medzi zrážajúcimi sa
% jadrami. Jadrá môžu by\softt{} taktiež spojené na určitý
% čas, ale po zrážke binárny charakter systému ostáva. Pri
% enrgiách nad touto binárnou energiou sa systém môže
% roztriešti\softt{}. Pre centrálne symetrické zrážky, tento
% systém môže utrpie\softt{} totálnu explóziu. Pre okrajové
% zrážky je často použity tzv. participant-spectator
% (účastník-divák) model. Nukleóny, ktoré v\softd{}aka
% geometrickým príčinam sa nezúčastňujú pri
% zrážkach  sú označené ako spectators. Nukleóny 
% v~prekrivajúcich častiach zrážajúcich sa jadier sú
% participants. Tuna prebieha ohrev a stlačenie pričom vzniká tzv.
% fireball, ktorý je bohatý na baryóny a tiež obsahuje ve\softl{}ký
% počet mezónov vytvorených v~zrážkach medzi terčíkom a 
% nalietavajúcim nukleónom.  Hustota energie sa očakáva okolo 2-3
% Gev/fm$^{3}$  a teplota okolo 200 MeV. Dúfa sa, že tieto podmienky
% budú dostatočne extrémne na vznik QGP.
% 
% Za účelom výskumu plazmy boli navrhnuté vysokoenergetické 
% zrážky \softt{}ažkých  iónov, kde pri ve\softl{}kom
% množstve interakcií nukleónov je šanca, že vzniknutý 
% fireball bude ma\softt{} dostatočnú teplotu a hustotu. Pritom energie
% zrážky sa zvyšuje so zvyšovaním nukleonového
% čísla A.
% 
% Akcelerácia \softt{}ažkých iónov sa stala možnou v~CERNe a 
% Brookhavene v~1986.  Urych\softl{}ované boli ióny, kde pomer
% protonového a nukleonového čísla bol Z/A=1/2. V~Brookhavene to
% boli kyslíkove a kremíkove zväzky urýchlené na 14.5 Gev/c
% na nukleón, zatia\softl{} čo v~CERNe kyslíkove a sírove
% ióny boli urýchlené do energií 200 GeV/c na nukleón.
% %\vspace*{10.5cm}
% 
% \begin{center}
%   \includegraphics*[bb=95 470 515 805,width=12cm]{ioncol.eps}
% \end{center}
% 
% \begin{center}
%   Obrázok 2.2: Nukleón nukleónova zrážka v~CMS sústave.
% \end{center}
% %obrazok z Barnesa z 21 str.
% 

\section{Evolúcia plazmy}
Ak sa QGP vytvorí v~zrážkach ťažkých iónov,
jeden z možných scenárov je, že vzniká ako je ukázané na obr. 2.2.
Tvar kuže\softl{}a je závislý na dráhach projektilových a
terčíkových  jadier. Očakáva sa, že doba
života fireballu   je okolo  10$^{-23}$~s [3]. Po tomto čase
teplota klesne pod kritickú hodnotu v~dôsledku expanzie a radiácie
častíc z~povrchu. Hmota fireballu potom začne
hadronizova\softt{} na nieko\softl{}ko sto mezónov a baryónov, ktoré 
%sa nakoniec prestanú zráža\softt{} a uletia, tieto potom môžme
potom môžeme 
registrovať v~detektoroch umiestnených v~rozumných vzdialenostiach
od cie\softl{}a.
\begin{center}
\includegraphics*[bb=120 355 510 725,width=12cm]{casprie.eps}
\end{center}
\begin{center}
  Obrázok 2.2: Ideálny časovo priestorový diagram evolúcie QGP.
\end{center}
%obrazok z Bayesa z 8 str.

\section{Príznaky QGP}
Ak sme schopní vytvori\softt{} podmienky na vznik plazmy, ako potom 
zistíme jej prí\-tom\-nos\softt{}? V~tejto časti prediskutujeme
rôzne experimentálne signály plazmy.

Zrážky \softt{}ažkých jadier nám prinajlepšom 
zabezpečia iba malý prechod cez hranice fázy
hadrónovej hmoty k~fáze plazmy a je dôležité  mať na
pamäti, že akáko\softl{}vek fáza plazmy vyprodukovaná
v~iónových zrážkach bude v prechodnom stave [5]. Mnoho
experimentálnych pozorovaní preto nebude reprezentova\softt{}
práve fázu plazmy, ale skôr akýsi post-plazmatický stupeň.
Vydeli\softt{} príspevok z~fázy plazmy nie je triviálna úloha. \\
\hspace*{0.45cm} Ako sa ukazuje, fireball by mal ma\softt{} krátku 
životnos\softt{}, sná\softd{} $\tau\sim$ 5-10 fm/c alebo okolo
10$^{-23}$~s, po ktorej teplota klesne pod kritickú hodnotu T$_{c}$
v\softd{}aka expanzii a radiačnému ochladeniu [6]. Hustota bude
stále vysoká, aj potom čo sa fáza zmení zo stavu plazmy, a tak
hadróny interagujú stále silne, až kým stredná
vzdialenos\softt{} častíc nebude vačšia ako dosah silnej
interakcie. A~práve tieto interakcie budú ma\softt{} za následok
\uv{znehodnotenie} možných príznakov QGP.

Takže, ak vôbec QGP vznikla, treba sa zamera\softt{} na také 
signály, ktoré sa  v~procese hadronizácie nemenia, to znamená
sústredi\softt{} sa na štúdium:
\begin{enumerate}
  \item[a)] takých častíc, ktoré neinteragujú silno
  \item[b)] kvantových čísel, ktoré sa nemenia v~silných 
interakciách
\end{enumerate}
Boli navrhnuté niektoré oblasti štúdia ako potenciálne 
užitočné indikátory existencie plazmy: 
\begin{enumerate}
  \item[1.] produkcia priamych fotónov a leptónov
\end{enumerate}
tento znak spadá do kategórie a)
\begin{enumerate}
  \item[2.] zvýšená produkcia podivných častíc 
\end{enumerate}
tento znak spadá do kategórie b) a
\begin{enumerate}
  \item[3.] potlačenie produkcie J/$\Psi$ mezónov.
\end{enumerate}
Všeobecne sa priznáva, že neexistuje jednoznačný signál, ktorý by bez pochýb
ukazoval na prítomnosť plazmy.

\subsection{Produkcia priamych fotónov a leptonóv}
Fotóny, ktoré vznikajú v~reakciách typu ako 
\begin{center}
$q \bar{q} \longrightarrow \gamma g$ 
\end{center}
\begin{center}
$g q \longrightarrow \gamma q$ 
\end{center}
a nie v~rozpadoch rezonancií, sa nazývajú priame fotóny. Tieto 
fotóny interagujú elektromagneticky, a nie silne, preto by z~QGP  mali
uniknúť bezo zmeny. Predpokladá sa, že fotóny by mali 
by\softt{} produkované v~intervale priečnych hybností $1 \leq
p_{T} \leq 3$ GeV/c [6]. Problém je však  v~tom, že fotóny
produkované v~QGP ve\softl{}mi \softt{}ažko rozlíšime od
fotónov  pochádzajúcich z~hadrónových rozpadov typu $ \pi^{0}
\longrightarrow \gamma \gamma $ alebo $ \eta \longrightarrow \gamma \gamma$ a
$\eta \longrightarrow 3\pi^{0}$, ktoré tvoria ve\softl{}ké pozadie.
Ukazuje sa, že viac ako polovica celkovej multiplicity pochádza práve
z~týchto hadrónových rozpadov. Takže detekcia priamych fotónov je
problematická v\softd{}aka ve\softl{}kému pozadiu.

O QGP by nám mohli poskytnúť informácie aj  leptonové páry
produkované v termálnych procesoch ako sú anihilácie kvark-antikvark 
\begin{center}
  $q\bar{q} \longrightarrow l^{+}l^{-}$.
\end{center}
Podobne ako priame fotóny tak aj páry leptón-antileptón nie sú ovplyvňované
silnými procesmi a môžu z plazmy voľne unikať. No aj v tomto prípade je
problém s vysokým pozadím pochádzajúcim z procesov ako je Drell-Yan, kde sa
$q\bar{q}$ anihilácia  vyskytuje aj pri interakciách hadrónov.


% \subsection{Produkcia
% dileptónov} Tak ako fotóny tak aj leptóny neinteragujú silne a
% môžu  prispieva\softt{} k~štúdiu QGP. Dileptóny sú
% produkované v~plazme cez kvark-antikvarkové elektromagnetické
% anihilácie. Ale znova tu nastáva problém s~vysokým pozadím
% pochadzajúcim od rozpadov excitovaných vektorových mezónov. 

\subsection{Potlačenie produkcie J/$\Psi$ a $\Psi^{'}$ mezónov}
Ako ďalší nástroj na hľadanie plazmy bola navrhnutá 
spektroskopia ťažkých vektorových mezónov, osobitne J/$\Psi
\longrightarrow \mu^{+} \mu^{-}$. J/$\Psi$ mezón je väzbový stav
pôvabného a antipôvabného kvarku $c\bar{c}$. Predpokladá sa, že
ak sa vytvorí QGP, tak pozorovaná produkcia J/$\Psi$ mezónu, by mala
byť nižšia ako produkcia v~bežných hadrónových
interakciách v~dôsledku mechanizmu farebného tienenia [7]. Silná
väzbová sila, ktorá navzájom viaže kvarky, je v~stave plazmy,
kde je veľká hustota pôvabných a antipôvabných kvarkov
vytvorených v~procesoch akými sú $q\bar{q} \rightarrow c\bar{c}$ a $gg
\rightarrow c\bar{c}$ tienená.
A~ak je polomer farebného tienenia menší ako veľkosť
J/$\Psi$, nemôže byť vytvorený väzbový stav $c\bar{c}$.
Veľkosť J/$\Psi$ väzbového stavu je okolo 0.5 fm, zatiaľ
čo polomer farebného tienenia  bol vypočítaný oveľa
menší. Pôvabné kvarky sa takto separujú a objavujú sa po
hadronizácii vo forme D mezónov.

Tiež sa predpokladá závislosť produkcie od $p_{T}$, keď
\uv{rýchly} J/$\Psi$  mezón unikne pred rozpustením v~plazme.
Zvýšená produkcia by mala byť výraznejšia pri
vyšších priečnych hybnostiach a mala by klesať 
s~klesajúcim $p_{T}$.

V~experimente NA38 skutočne pozorovali potlačenie produkcie
J/$\Psi$  v~zrážkach ťažkých iónov kyslík-urán a
síra-urán využívajúc leptónovú spektroskopiu na
identifikáciu J/$\Psi$ [8]. Predpovedaná $p_{T}$ závislosť bola
tiež pozorovaná. Hoci sú tieto výsledky zhodné s~predpoveďou
produkcie QGP, dajú sa tiež interpretovať aj bez prítomnosti
QGP, kde napr. J/$\Psi$ interaguje v~hustom hadrónovom plyne,
bez ohľadu na to, či plazma vznikla alebo nie. Takže tieto
výsledky samé o~sebe nemôžu byť brané ako dôkaz vzniku
plazmy.

Za predpokladu vzniku QGP bola predpovedaná znížená
pravdepodobnosť produkcie  $\Psi^{'}$ mezónu v~porovnaní
s~J/$\Psi$.
Experimenty so zrážkami p$\,$-A  ukazujú pomer medzi $\Psi^{'}$ a
J/$\Psi$ produkciou ako funkciu hustoty energie zrážky. Jasný pokles
tohoto pomeru je pozorovaný pri zvyšovaní hustoty energie [8].

\newpage \subsection{Zvýšená produkcia podivných častíc}
Doba života horúceho fireballu produkovaného v~zrážkach ťažkých
iónov je príliš krátka na to, aby mala
nejaký význam pre slabé reakcie, kde doba života môže byť
až $\sim 10^{15}$ krát vačšia. Preto vyprodukovaná
podivnosť môže byť zničená len anihiláciou podivného
kvarku a antikvarku. Keďže tento proces je dosť
nepravdepodobný, množstvo podivných častíc  pozorovaných
po hadronizácii je očakávaným dobrým indikátorom množstva
podivnosti  v~plazme, ak sa takáto fáza objavila.
Bolo navrhnuté, že produkcia podivných častíc v~QGP by mala
byť väčšia ako v~normálnych hadrónových reakciách [9].
Hlavné dôvody, pre ktoré sa očakáva zvýšená produkcia
podivnosti v~QGP sú:

\subsubsection{Produkcia častíc v oblasti nad prahovou energiou} 
Prahová energia na produkciu podivnosti v~QGP je množstvo
energie potrebnej k~produkcii páru $s\bar{s}$, na čo je potrebná
dvojnásobná  hmotnosť podivného kvarku $2m_{S} \sim 300$ MeV.
Pretože očakávaná teplota plazmy je okolo 200 MeV, je
energeticky výhodnejšie produkovať páry $s\bar{s}$ z~$gg$ a
$q\bar{q}$ anihilácií v~plazme.

V~hadrónových reakciách je však produkcia podivnosti vo forme
podivných hadrónov, ako sú napr:
\begin{center}
 $ K^{0}=(d\bar{s}), K^{-}=(\bar{u}s), \Lambda=(uds), \Sigma=(qqs), 
 \Xi=(qss) \: a \: \Omega=(sss)$ 
\end{center}
 a ich  príslušných antičastíc. Jednou z~reakcií, pri
ktorej sa 
 pro\-du\-ku\-jú po\-di\-vné častice s~najnižšiou energiou
je 
\begin{center}
 $p + p \longrightarrow p + K^{+} + \Lambda$ 
\end{center}
s~potrebnou prahovou CMS energiou okolo 570 MeV. Produkcia dvojice 
podivných anti\-ba\-ryónov je energeticky ešte
náročnejšia napr. 
\begin{center}
  $p + p \longrightarrow p + p + \Lambda + \bar{\Lambda}$
\end{center}
s~potrebnou prahovou energiou okolo 2.23 Gev [10].

Preto nie je prekvapujúce, že zvýšená produkcia podivných
hadrónov a osobitne podivných antibaryónov je očakávaným
dobrým príznakom tvorby QGP. Tiež sa predpokladá ešte viac
zvýšená produkcia multipodivných baryónov a antibaryónov oproti
produkcii týchto častíc v~hadrónových interakciách, kde je
energetická náročnosť značne veľká [11].
  
\subsubsection{Pauliho princíp}
V~QGP produkovanej zrážkou ťažkých iónov bude
veľký počet $u$ a $d$ kvarkov pochadzajúcich
z~počiatočných jadier týchto iónov. Keďže všetky kvarky sú
fermióny, produkcia ďalších $u$ a $d$ kvarkov ako párov
$u\bar{u}$ a $d\bar{d}$ gluón-gluónovou fúziou, bude znížená
v~dôsledku Pauliho vylučovacieho princípu. Takto bude
podporovaná produkcia párov podivný kvark-antikvark.  

\subsubsection{Veľký počet gluónov}
Podivné kvark-antikvarkové páry môžu byť vytvorené
gluón-gluónovou fúziou alebo anihiláciou kvark-antikvarkov.
% Obr. 2.4 ukazuje Feynmanové diagramy produkcie $s\bar{s}$
%párov gluónmi a ľahkými kvarkami. 
Ukazuje sa, že dominantným
procesom produkcie $s\bar{s}$ párov je gluón-gluónova fúzia [12].
%\vspace*{11cm}
%\begin{center}
%  Obrázok 2.4: Feynmanové diagramy z~produkciou $s\bar{s}$ kvarkov.
%\end{center}
%obrazok z Evansa z 18 str.
Nehľadiac na životnosť plazmy, množstvo podivných
častíc v~plazme bude naďalej narastať, až kým bude
množstvo podivných párov kvark-antikvark tak veľké, že 
pomer produkcií podivných kvarkov-antikvarkov  sa vyrovná anihilácii:
\begin{center}
  $\sigma(gg \rightarrow s\bar{s}) + \sigma(q\bar{q} \rightarrow s\bar{s}) = 
\sigma(s\bar{s} \rightarrow gg) + \sigma(s\bar{s} \rightarrow q\bar{q})$.
\end{center}
V~tomto štádiu ostane množstvo podivných častíc
konštantné v~rovnováhe. Čas potrebný nato, aby v~QGP nastal
stav rovnováhy, bol vypočítaný pri teplotách okolo 200 MeV  na
2-3$\times 10^{-23}$ s [12]. Minimálna doba života fireballu bola
vypočítaná na  3$\times 10^{-23}$ s~(10 fm/c). Tento odhad
je založený na predpoklade, že fireball bude mať v
priemere približne 10 fm.

V prípade ak sa vytvorí QGP, tak
tento čas je dostatočne dlhý nato, aby produkcia podivných
častíc dosiahla stav nasýtenia.  Keďže prakticky
žiadne podivné kvarky nie sú prítomné pred zrážkou,
množstvo podivných kvarkov sa značne zvýši do dosiahnutia
rovnováhy. V~prípade neprítomnosti QGP, ale len hustého
hadrónového plynu, rýchlosť produkcie podivných častíc
by bola 10-30 $\times$ nižšia a je nepravdepodobné, aby sa dosiahol
stav rovnováhy [13].

\subsection{Globálne príznaky}
Okrem doteraz spomínaných špecifických príznakov QGP,
existujú aj globálne príznaky, ktoré teraz v~stručnosti
pripomeniem. 
\subsubsection{Fázový diagram}
Klasický fázový prechod môže byť opísaný ako plató
nasledované druhým vzostupom na grafe relevantných termodynamických
premenných, napr. v~p-V diagrame fázového prechodu voda-para.
Intuitívnym prístupom navrhnutým Van Hoveom je skúšať
niečo podobné v~spektre produkovanom v~interakciách ťažkých
iónov [14]. Priečna hybnosť $p_{T}$  úzko
súvisí s~teplotou T a hustotou energie $\epsilon$ a môže
súvisieť s~počtom produkovaných častíc na jednotku
rapidity.

\vspace*{6cm}
%\begin{center}
\hspace*{-2.3cm}
\input{ptode.pic}
%\end{center}
\vspace*{-8cm}
\begin{center}
  Obrázok 2.3: Závislosť $p_{T}$ od hustoty energie $\epsilon$.
\end{center}
%obrazok uz hotovy
% \subsubsection{Interferometria}
% Táto myšlienka je založená na meraní polemeru hviezd
% v~astrofyzike, kde nositeľom informácie je fotón . V~subjadrovej
% fyzike vymeníme fotóny za pár nabitých piónov. Táto metóda
% nám potom umožní merať rozmer zóny v~stave, keď horúca
% a hustá hmota sa mení na hadrónové častice, najmä
% pióny. 

\subsubsection{Rozdelenie priečnych hybností a hmotností}
Emisia častíc z~horúceho fireballu v~tepelnej rovnováhe
môže byť vyjadrená  vzťahom
\begin{center}
 $\frac{1}{p_{T}}\frac{dN}{p_{T}} =
 \frac{1}{T}\sqrt{m_{T}}\exp-\frac{m_{T}}{T}$.
\end{center}
Účinný prierez a priečnu hmotnosť $m_{T}$ získame
z~experimentu. Teplota sa potom získa fitovaním spektra formulou v
súlade s Hagedornovou rovnicou. Takto získaná teplota v~experimentoch
s~ťažkými iónmi sa pohybuje v~rozmedzi 150-200 MeV, čo by
už mala byť dostatočná teplota na prechod do fázy QGP [15].

\section{Zhrnutie}
Predpoklad QCD, že pri dostatočnej hustote a teplote sa môže
objaviť fázový prechod z~hadrónovej hmoty do QGP je v~centre
pozornosti fyzikov vysokých energií aj astrofyzikov. Niektoré znaky QGP boli
navrhnuté teoretikmi a od roku 1986 sa experimentátori v~CERNe a BNL
pokúšajú vytvoriť  podmienky pre vytvorenie QGP. Experiment
WA94 nadväzuje na prácu WA85, kde bolo pozorované dvojnásobné
zvýšenie produkcie podivných častíc v~centrálnych SW
zrážkach  v~porovnaní s~pW interakciami [16].
% \begin{table}[h]
% \begin{center}
%   \begin{tabular}{|cc|}
%    \hline Častica & ${\normalsize\frac{vytazok_{SW}}
% {vytazok_{pW}}}$ \\
%    \hline $K^{0}$ & 1.33 $\pm$ 0.13 \\
%           $\Lambda$ & 2.02 $\pm$ 0.20 \\
%           $\bar{\Lambda}$ & 1.64 $\pm$ 0.16 \\
%    \hline        
%   \end{tabular}
% \end{center}
% \caption{Výsledky experimentu WA85}
% \end{table}
