\chapter{Úvod}
Na začiatku 80-tych rokov bola predpovedaná možnosť
vytvorenia novej fázy látky, kvark-gluónovej plazmy (QGP), v~zrážkach
ultra-relativistických ťažkých iónov. To viedlo k~sériam
experimentov s~ťažkými iónmi začínajúcich v~roku 1986
v~Brookhavene na Altering gradient synchrotron (AGS) a~v~CERNe na Super
proton synchrotron (SPS). Jeden z~týchto experimentov bol aj WA94, ktorý
sa zaoberal produkciou podivných častíc v~zrážkach jadier sí
ra-síra pri energiach 200 GeV/c na nukleón. QGP bola jednou z
predpovedí kvantovovej chromodynamiky (QCD).

\section{QCD, farba a uväznenie kvarkov}
% QCD opisuje silné interakcie medzi kvarkami a nehmotnými gluónmi.
% Poznáme 6 kvarkov rôzných \uv{vôni} ako ukazuje tab. \ref{chute}.
% Všetky kvarky sú fermióny, t.j. majú spin $\frac{1}{2}$.
% \begin{table}[h]
% \begin{center}
% \begin{tabular}{|lcc|}
% \hline {\bf kvark} & {\bf Q [e]} & {\bf m [MeV]} \\
% \hline u=\uv{up} & +2/3 & 5 \\
% d=\uv{down} & -1/3 & 9 \\
% s=\uv{strange} & -1/3 & 170 \\
% c=\uv{charmed} & +2/3 & 1400 \\
% b=\uv{bottom} & -1/3 & 4400 \\
% t=\uv{top} & +2/3 & 170 000 \\
% \hline
% \end{tabular}
% \end{center}
% \caption{vône\ kvarkov}\label{chute}
% \end{table} \\
V~roku 1951 bola pozorovaná rezonancia $\Delta^{++}$, kde sa tri kvarky
majú nachádzať v~rovnakom spinovom stave, čo Fermiho štatistika
zakazuje. Jedna možnosť riešenia viedla
k~zavedeniu pojmu \uv{farby}. Každej vôni kvarku sa priradili tri
rôzne farebné stavy tak, aby výsledný objekt bol \uv{bezfarebný}.
To znamená, že kvarky sa transformujú v~súlade s~novou farebnou
$SU_c(3)$ grupou a pozorované hadróny sú farebné singlety, t.j javia
sa ako bezfarebné, \uv{biele} objekty. Na rozdiel od kvantovej
elektrodynamiky, kde sú fotóny elektricky neutrálne, kvantová
chromodynamika predpovedá farebný náboj gluónov. Okrem kvarkov, ktoré
interagujú prostredníctvom výmenných gluónov, interagujú aj samotné gluóny.
Tento fakt vyplýva z neabelovskej grupy $SU_c(3)$, ktorá sa používa na popis
rotácii vo farebnom priestore [1]. Gluóny existujú v ôsmich farebných
stavoch a prenášajú tok farebného náboja od jedného kvarku k druhému. 
Siločiary interakcie medzi dvoma kvarkami sú
uzavreté v trubici valcovitého tvaru na rozdiel od elektromagnetickej  
interakcie, kde sa rozbiehajú do všetkých smerov. V tejto trubici
je konštantná hustota energie na jednotku dĺžky [1], preto potenciál medzi
kvarkom a antikvarkom bude lineárne rásť so vzdialenosťou a má tvar
\begin{center} 
{\large $V_{S}(r) \sim -\frac{\alpha_{S}}{r} + \gamma r,$ }
\end{center}
kde $\alpha_{S}$ je konštanta silnej interakcie, $r$ vzdialenosť medzi
kvarkom a antikvarkom. Lineárny člen implikuje, že na uvoľnenie
kvarku treba nekonečne veľkú energiu. To je dôvod, prečo sú kvarky za
normálnych podmienok uväznené, t.j. pozorujú sa iba ich viazané stavy,
kde je celkový farebný náboj nulový. Tieto \uv{biele} objekty sú hadróny.


\subsection{Asymptotická sloboda}
V~kvantovej elektrodynamike (QED) elektrón vo vákuu sám o~sebe nemô\v
ze existovať, ale nepretržite  emituje a absorbuje virtuálne
fotóny,  z~ktorých niektoré môžu produkovať elektrón-pozitrónove
páry. Tento efekt spôsobuje, že v~akomkoľvek danom okamihu bude
elektrón obklopený určitým  počtom elektrón-pozitrónových
párov , ktoré sa usporiadajú tak, že
pozi\-tróny smerujú  k~elektrónu. Náboj daného elektrónu bude
tienený týmto nabitým oblakom, ktorý ho obklopuje, ako ukazuje obr.
1.1a [4]. Veľkosť náboja, ktorý meriame, bude závisieť od
energie našej  sondujúcej častice, od vzdialenosti najväč\v
sieho  priblíženia. 

V~QCD sa objavuje podobný proces
s~kvarkom, ktorý emituje gluóny. Tieto gluóny, však na rozdiel od
fotónov, môžu interagovať aj medzi sebou a tak vytvárať
gluónove slučky. Teda napr. červený kvark je uprednostnene
obklopený práve červenými nábojmi a tak nízkoenergetická sonda
bude merať farebný náboj, ktorý je vyšš\'\i\, ako samotný
holý farebný náboj kvarku, obr. 1.1b. So zvyšovaním energie
sondy sa asymptoticky blížime  k~veľkosti holého farebného
náboja, no pri takýchto malých vzdialenostiach je farebná sila medzi
dvoma kvarkami veľmi malá a kvarky dosiahnu stav, v~ktorom sa viac
menej sparávajú ako voľné neinte\-ragujúce častice. Tento stav
nazývame stavom asymptotickej slobody.

Pri veľmi vysokých teplotách a hustotách hadrónovej hmoty
by táto hadrónová hmota mohla prejsť do fázy, v~ktorej by
sa hadróny \uv{rozliali} na asymptotický voľný plyn kvarkov a
gluónov, kde by sa kvarky v rámci tohto priestoru mohli voľne  pohybovať.
Táto predpokladaná fáza je stav kvark-gluónovej plazmy.

\newpage

\begin{center}
\includegraphics*[bb=108 210 530 820,width=14.4cm]{tienenie.eps}
\end{center}

\begin{center}
Obrázok 1.1: Tienenie pre a) elektrický náboj $e^{-}$ a b) farebný 
náboj q.
\end{center}
%obrazok z Bayesa z 3 str.

