\chapter{Rekonštrukcia $K^{0}$ rozpadov}
Experiment WA94 bol navrhnutý za účelom skúmania QGP, konkrétne pozorovaním
zvýšenej produkcie podivných častíc $K^{0}, K^{\pm}, \Lambda,
\overline{\Lambda}, \Xi^{-}, \overline{\Xi^{-}}$. Táto práca sa zaoberá
štúdiom produkcie $K^{0}_{S}$ mezónov. Neutrálne hadróny môžeme identifikovať
na základe ich rozpadov na nabité častice: \\
\begin{center}
  $K^{0}_{S} \longrightarrow \pi^{+} + \pi^{-}$ \\
\end{center}
\begin{center}
  $\Lambda \longrightarrow p + \pi^{-}$. \\
\end{center}
Teda znakom prítomnosti $K^{0}_{S}$ mezónu sú dve stopy opačne nabitých
častíc, $\pi^{+}$ a $\pi^{-}$ mezónov. \\
\vspace*{-5cm} \hspace*{-1cm} \input{pic/rozpad.pic}

\vspace*{-3cm}
\begin{center}
  Obrázok 4.1: Rozpad $K^{0}_{S}$ mezónu.
\end{center}
\section{Rekonštrukcia $V^{0}$}
Ak už máme experimentálne  dáta spracované pomocou programu TRIDENT, ktorý
rekonštruuje  dráhy nabitých častíc z jednotlivých \uv{hitov} v~danej
komore, tak potom tieto zrekonštruované dráhy vstupujú do programu STRIPV0.
Tento program hľadá dve opačne nabité dráhy pretínajúce sa  v~bode dostatočne
vzdialenom od terčíka, aby nedochádzalo k~zámenam s~nabitými časticami
pochádzajúcimi priamo z~terčíka. Vykonáva aj určité orezania (cuts)
na vyprodukovanie tzv. DST (data summary tape) s~$V^{0}$ kandidátmi:

\begin{itemize}
\item{ Ak je vzdialenosť dráh dvoch opačne nabitých častíc  v~mieste ich
  najväčšieho priblíženia väčšia ako 1 cm, tak takýto pár sa
  z~ďalšej analýzy  vylučuje. Ak nie, tak vertex $V^{0}$ je potom bod ležiaci
  v~polovici tohto miesta.}
\item{X-ová súradnica vertexu $V^{0}$ leží v~intervale
  $-115cm<x_{\Omega}<71cm$. Spodná hranica tejto súradnice je taká, pretože
  akceptancia je veľmi nízka pre $V^{0}$ rozpadajúce sa ďaleko od MWPC. Horná
  hranica je daná polohou prvej komory A1.}
\item{Tak isto sa vyžaduje, aby priečna zložka hybností rozpadajúcich
  sa dráh vzhľadom k~smeru pohybu $V^{0}$, $q_{T}$ bola menšia ako $0.3\:
  GeV/c$. Skutočné $V^{0}$ majú $q_{T}<0.2\: GeV/c$.}
\item{Celková hybnosť $V^{0}$ musí byť väčšia ako 6.5 GeV/c, čo je
  minimálna hybnosť pre takéto rozpady, aby boli merateľné v~Omega A~komorách.
  Typické hodnoty hybností $K^{0}$ registrovaných v~komorách sú okolo 15
  GeV/c.}
\item{Rozpadové produkty musia preletieť všetkými siedmimi MWPC a na
  rekonštrukciu dráhy sa použili aspoň štyri priestorové body.}
\end{itemize}
\section{Identifikácia $K^{0}$ mezónov}
Obr. 4.1 znázorňuje $K^{0}$ rozpad, základné
geometrické parametre pre rozpad: vzdialenosť  najmenšieho priblíženia dvoch
rozpadajúcich dráh $d$, pozíciu vrcholu a impakt parameter, o~ktorom
pojednám neskôr.

Vrchol rozpadu $K^{0}$ by sa mal nachádzať v~intervale
$-100cm<x_{\Omega}<71cm$. Pre uistenie, že $K^{0}$ pochádza
z~terčíka, musí byť  uhol $\vartheta$ medzi priamkou určenou polohou $V^{0}$
a terčíkom a vektorom hybností menší ako $0.75^{\circ}$.

Vhodným nástrojom  na identifikáciu častíc sa ukázal  graf
Armenterosa--Podo\-lanského.
Na obr. 4.2a je znázornený rozpad $V^{0}$ v~laboratórnej sústave na častice 
s~kladným (+) a záporným (-)  nábojom s~pozdĺžnou hybnosťou  $p_{L}$ v~smere
letu $V^{0}$ a priečnou hybnosťou $p_{T}$. V~CMS sústave, obr. 4.2b,  zložky
hybností kladnej (+) častice majú tvar
\begin{center}
  $ p_{L}^{*}(+)=p^{*}\cos\theta^{*}$ \\  
  $ p_{T}^{*}(+)=p^{*}\sin\theta^{*}$
\end{center}
a energia 
\begin{center}
  $ E^{*}(+)=\sqrt{p^{{*}^{2}} + m^{2}(+)}$.
\end{center}

\newpage
\begin{center}
Na obr. 4.2a je znázornený rozpad $V^{0}$ v~laboratórnej sústave na častice 
s~kladným (+) a záporným (-)  nábojom s~pozdĺžnou hybnosťou  $p_{L}$ v~smere
letu $V^{0}$ a priečnou hybnosťou $p_{T}$. V~CMS sústave, obr. 4.2b,  zložky
hybností kladnej (+) častice majú tvar
\begin{center}
  $ p_{L}^{*}(+)=p^{*}\cos\theta^{*}$ \\  
  $ p_{T}^{*}(+)=p^{*}\sin\theta^{*}$
\end{center}
a energia 
\begin{center}
  $ E^{*}(+)=\sqrt{p^{{*}^{2}} + m^{2}(+)}$.
\end{center}

\newpage
\begin{center}
Na obr. 4.2a je znázornený rozpad $V^{0}$ v~laboratórnej sústave na častice 
s~kladným (+) a záporným (-)  nábojom s~pozdĺžnou hybnosťou  $p_{L}$ v~smere
letu $V^{0}$ a priečnou hybnosťou $p_{T}$. V~CMS sústave, obr. 4.2b,  zložky
hybností kladnej (+) častice majú tvar
\begin{center}
  $ p_{L}^{*}(+)=p^{*}\cos\theta^{*}$ \\  
  $ p_{T}^{*}(+)=p^{*}\sin\theta^{*}$
\end{center}
a energia 
\begin{center}
  $ E^{*}(+)=\sqrt{p^{{*}^{2}} + m^{2}(+)}$.
\end{center}

\newpage
\begin{center}
\input{alfa.pic}
\end{center}
\vspace{-6cm}
\begin{center}
  Obrázok 4.2: Schématické znázornenie $V^{0}$ rozpadu.
\end{center}
Transformácia do laboratórnej sústavy pomocou Lorentzovych transformácií má
v~maticovom zápise  tvar
\begin{center}
  $$ \Biggl( \begin{array}{c}
     E \\
     p_{L} \end{array} \Biggr) = \Biggl( \begin{array}{cc} \gamma &
\gamma\beta \\ \gamma\beta & \gamma \end{array} \Biggr) \Biggl(
\begin{array}{c} E^{*} \\ p_{L}^{*} \end{array} \Biggr), \hspace{1cm}
p_{T}^{*}= p_{T} $$
\end{center}
kde $\gamma=1/\sqrt{1-\beta^{2}}$. Transformáciou dostávame 
\begin{center}
  $E^{Lab}(+)=\gamma E^{*}(+) + \gamma\beta p_{L}^{*}(+)$ \\
  $p_{L}^{Lab}(+)=\gamma p_{L}^{*}(+) + \gamma\beta E^{*}(+) = \gamma
p^{*}\cos
  \theta^{*} + \gamma\beta E^{*}(+)$ \\
  $p_{T}^{Lab}(+)=p_{T}^{*}(+)= p^{*}\sin\theta^{*},$
\end{center}
to znamená, že pozdĺžná hybnosť pre kladnú (+) a zápornú (-) časticu má tvar
\begin{center}
  $p_{L}^{Lab}(+)= \gamma p^{*}\cos  \theta^{*} + \gamma\beta E^{*}(+)$ \\
  $p_{L}^{Lab}(-)= -\gamma p^{*}\cos  \theta^{*} + \gamma\beta E^{*}(-)$
\end{center}
potom dostávame 
\begin{center}
  $p_{L}^{Lab}(+)-p_{L}^{Lab}(-)=2\gamma p^{*}\cos\theta^{*} +\beta\gamma
(E^{*}(+)-E^{*}(-)) $ \\
  $p_{L}^{Lab}(+)+p_{L}^{Lab}(+)= p_{L}^{Lab}=\beta\gamma m_{{V}^{0}}$.
\end{center}
Armenteros--Podolanského $\alpha$ je definovaná ako
\begin{center}
 $\alpha={\Large\frac{p_{L}^{Lab}(+)-p_{L}^{Lab}(-)}{p_{L}^{Lab}(+)+p_{L}^{Lab}(+)}}
= {\Large\frac{2p^{*}\cos\theta^{*}}{\beta m_{{V}^{0}}}}+{\Large\frac{E^{*}(+)-E^{*}(-)}
{m_{{V}^{0}}}} = \zeta\cos\theta^{*} + \varphi$
\end{center}
to znamená 
\begin{center}
  $\cos\theta^{*}={\Large\frac{\alpha-\varphi}{\zeta}}$ \hspace{1cm} $\sin\theta^{*}
= {\Large\frac{p_{T}^{*}}{p^{*}}}$ \\
  $\cos^{2}\theta^{*} + \sin^{2}\theta^{*} =1={\Large \bigl(\frac{\alpha-\varphi}
{\zeta} \bigr)^{2}} + {\Large\bigl(\frac{p_{T}^{*}}{p^{*}} \bigr)^{2}}$.
\end{center}
Tento posledný výraz je rovnica elipsy so stredom v bode $(\varphi,0)$ a
poloosmi v~smere $\alpha$ veľkosti $\zeta$ a $p_{T}$ veľkosti $p^{*}$. V tab.
4.1. sú tieto hodnoty pre rôzne $V^{0}$ a na obr. 4.3  sú znázornené elipsy
pre rozpady $\Lambda$, $\overline{\Lambda}$ a $K^{0}$.

\vspace*{0.5cm}
\begin{tabular}{|l|c|c|c|c|c|}
\hline
Rozpad & $\zeta$ & $p^{*}[GeV/c]$ & $\alpha_{min}$ & $\alpha_{max}$ & 
$\varphi$ \\ \hline
$K^{0}_{S}\rightarrow\pi^{+}\pi^{-}$ & 0.8282 & 0.206 & -0.8282 & +0.8282 &
0 \\
$\Lambda\rightarrow p\pi^{-}$ & 0.179 & 0.101 & +0.515 & +0.873 & 0.694 \\
$\overline{\Lambda}\rightarrow \overline{p}\pi^{+}$ & 0.179 & 0.101 &
-0.873 & -0.515 & -0.694 \\ \hline  
\end{tabular}
\begin{center}
  Tabuľka 4.1: Armenterovské premenné.
\end{center}

\end{center}
\vspace{-6cm}
\begin{center}
  Obrázok 4.2: Schématické znázornenie $V^{0}$ rozpadu.
\end{center}
Transformácia do laboratórnej sústavy pomocou Lorentzovych transformácií má
v~maticovom zápise  tvar
\begin{center}
  $$ \Biggl( \begin{array}{c}
     E \\
     p_{L} \end{array} \Biggr) = \Biggl( \begin{array}{cc} \gamma &
\gamma\beta \\ \gamma\beta & \gamma \end{array} \Biggr) \Biggl(
\begin{array}{c} E^{*} \\ p_{L}^{*} \end{array} \Biggr), \hspace{1cm}
p_{T}^{*}= p_{T} $$
\end{center}
kde $\gamma=1/\sqrt{1-\beta^{2}}$. Transformáciou dostávame 
\begin{center}
  $E^{Lab}(+)=\gamma E^{*}(+) + \gamma\beta p_{L}^{*}(+)$ \\
  $p_{L}^{Lab}(+)=\gamma p_{L}^{*}(+) + \gamma\beta E^{*}(+) = \gamma
p^{*}\cos
  \theta^{*} + \gamma\beta E^{*}(+)$ \\
  $p_{T}^{Lab}(+)=p_{T}^{*}(+)= p^{*}\sin\theta^{*},$
\end{center}
to znamená, že pozdĺžná hybnosť pre kladnú (+) a zápornú (-) časticu má tvar
\begin{center}
  $p_{L}^{Lab}(+)= \gamma p^{*}\cos  \theta^{*} + \gamma\beta E^{*}(+)$ \\
  $p_{L}^{Lab}(-)= -\gamma p^{*}\cos  \theta^{*} + \gamma\beta E^{*}(-)$
\end{center}
potom dostávame 
\begin{center}
  $p_{L}^{Lab}(+)-p_{L}^{Lab}(-)=2\gamma p^{*}\cos\theta^{*} +\beta\gamma
(E^{*}(+)-E^{*}(-)) $ \\
  $p_{L}^{Lab}(+)+p_{L}^{Lab}(+)= p_{L}^{Lab}=\beta\gamma m_{{V}^{0}}$.
\end{center}
Armenteros--Podolanského $\alpha$ je definovaná ako
\begin{center}
 $\alpha={\Large\frac{p_{L}^{Lab}(+)-p_{L}^{Lab}(-)}{p_{L}^{Lab}(+)+p_{L}^{Lab}(+)}}
= {\Large\frac{2p^{*}\cos\theta^{*}}{\beta m_{{V}^{0}}}}+{\Large\frac{E^{*}(+)-E^{*}(-)}
{m_{{V}^{0}}}} = \zeta\cos\theta^{*} + \varphi$
\end{center}
to znamená 
\begin{center}
  $\cos\theta^{*}={\Large\frac{\alpha-\varphi}{\zeta}}$ \hspace{1cm} $\sin\theta^{*}
= {\Large\frac{p_{T}^{*}}{p^{*}}}$ \\
  $\cos^{2}\theta^{*} + \sin^{2}\theta^{*} =1={\Large \bigl(\frac{\alpha-\varphi}
{\zeta} \bigr)^{2}} + {\Large\bigl(\frac{p_{T}^{*}}{p^{*}} \bigr)^{2}}$.
\end{center}
Tento posledný výraz je rovnica elipsy so stredom v bode $(\varphi,0)$ a
poloosmi v~smere $\alpha$ veľkosti $\zeta$ a $p_{T}$ veľkosti $p^{*}$. V tab.
4.1. sú tieto hodnoty pre rôzne $V^{0}$ a na obr. 4.3  sú znázornené elipsy
pre rozpady $\Lambda$, $\overline{\Lambda}$ a $K^{0}$.

\vspace*{0.5cm}
\begin{tabular}{|l|c|c|c|c|c|}
\hline
Rozpad & $\zeta$ & $p^{*}[GeV/c]$ & $\alpha_{min}$ & $\alpha_{max}$ & 
$\varphi$ \\ \hline
$K^{0}_{S}\rightarrow\pi^{+}\pi^{-}$ & 0.8282 & 0.206 & -0.8282 & +0.8282 &
0 \\
$\Lambda\rightarrow p\pi^{-}$ & 0.179 & 0.101 & +0.515 & +0.873 & 0.694 \\
$\overline{\Lambda}\rightarrow \overline{p}\pi^{+}$ & 0.179 & 0.101 &
-0.873 & -0.515 & -0.694 \\ \hline  
\end{tabular}
\begin{center}
  Tabuľka 4.1: Armenterovské premenné.
\end{center}

\end{center}
\vspace{-6cm}
\begin{center}
  Obrázok 4.2: Schématické znázornenie $V^{0}$ rozpadu.
\end{center}
Transformácia do laboratórnej sústavy pomocou Lorentzovych transformácií má
v~maticovom zápise  tvar
\begin{center}
  $$ \Biggl( \begin{array}{c}
     E \\
     p_{L} \end{array} \Biggr) = \Biggl( \begin{array}{cc} \gamma &
\gamma\beta \\ \gamma\beta & \gamma \end{array} \Biggr) \Biggl(
\begin{array}{c} E^{*} \\ p_{L}^{*} \end{array} \Biggr), \hspace{1cm}
p_{T}^{*}= p_{T} $$
\end{center}
kde $\gamma=1/\sqrt{1-\beta^{2}}$. Transformáciou dostávame 
\begin{center}
  $E^{Lab}(+)=\gamma E^{*}(+) + \gamma\beta p_{L}^{*}(+)$ \\
  $p_{L}^{Lab}(+)=\gamma p_{L}^{*}(+) + \gamma\beta E^{*}(+) = \gamma
p^{*}\cos
  \theta^{*} + \gamma\beta E^{*}(+)$ \\
  $p_{T}^{Lab}(+)=p_{T}^{*}(+)= p^{*}\sin\theta^{*},$
\end{center}
to znamená, že pozdĺžná hybnosť pre kladnú (+) a zápornú (-) časticu má tvar
\begin{center}
  $p_{L}^{Lab}(+)= \gamma p^{*}\cos  \theta^{*} + \gamma\beta E^{*}(+)$ \\
  $p_{L}^{Lab}(-)= -\gamma p^{*}\cos  \theta^{*} + \gamma\beta E^{*}(-)$
\end{center}
potom dostávame 
\begin{center}
  $p_{L}^{Lab}(+)-p_{L}^{Lab}(-)=2\gamma p^{*}\cos\theta^{*} +\beta\gamma
(E^{*}(+)-E^{*}(-)) $ \\
  $p_{L}^{Lab}(+)+p_{L}^{Lab}(+)= p_{L}^{Lab}=\beta\gamma m_{{V}^{0}}$.
\end{center}
Armenteros--Podolanského $\alpha$ je definovaná ako
\begin{center}
 $\alpha={\Large\frac{p_{L}^{Lab}(+)-p_{L}^{Lab}(-)}{p_{L}^{Lab}(+)+p_{L}^{Lab}(+)}}
= {\Large\frac{2p^{*}\cos\theta^{*}}{\beta m_{{V}^{0}}}}+{\Large\frac{E^{*}(+)-E^{*}(-)}
{m_{{V}^{0}}}} = \zeta\cos\theta^{*} + \varphi$
\end{center}
to znamená 
\begin{center}
  $\cos\theta^{*}={\Large\frac{\alpha-\varphi}{\zeta}}$ \hspace{1cm} $\sin\theta^{*}
= {\Large\frac{p_{T}^{*}}{p^{*}}}$ \\
  $\cos^{2}\theta^{*} + \sin^{2}\theta^{*} =1={\Large \bigl(\frac{\alpha-\varphi}
{\zeta} \bigr)^{2}} + {\Large\bigl(\frac{p_{T}^{*}}{p^{*}} \bigr)^{2}}$.
\end{center}
Tento posledný výraz je rovnica elipsy so stredom v bode $(\varphi,0)$ a
poloosmi v~smere $\alpha$ veľkosti $\zeta$ a $p_{T}$ veľkosti $p^{*}$. V tab.
4.1. sú tieto hodnoty pre rôzne $V^{0}$ a na obr. 4.3  sú znázornené elipsy
pre rozpady $\Lambda$, $\overline{\Lambda}$ a $K^{0}$.

\vspace*{0.5cm}
\begin{tabular}{|l|c|c|c|c|c|}
\hline
Rozpad & $\zeta$ & $p^{*}[GeV/c]$ & $\alpha_{min}$ & $\alpha_{max}$ & 
$\varphi$ \\ \hline
$K^{0}_{S}\rightarrow\pi^{+}\pi^{-}$ & 0.8282 & 0.206 & -0.8282 & +0.8282 &
0 \\
$\Lambda\rightarrow p\pi^{-}$ & 0.179 & 0.101 & +0.515 & +0.873 & 0.694 \\
$\overline{\Lambda}\rightarrow \overline{p}\pi^{+}$ & 0.179 & 0.101 &
-0.873 & -0.515 & -0.694 \\ \hline  
\end{tabular}
\begin{center}
  Tabuľka 4.1: Armenterovské premenné.
\end{center}

%  Tri zložky hybnosti rozpadových
% produktov $V^{0}$ prepočítame na priečne $q_{T}$ a pozdĺžne $q_{L}$ zložky
% vzhľadom k~smeru letu $V^{0}$. Pozdĺžne zložky kladných rozpadových produktov
% označíme $q_{L}^{+}$ a záporných produktov $q_{L}^{-}$. Zadefinujeme veličinu
% $\alpha$ ako:
%
% \begin{center}
%   $\alpha=\frac{q_{L}^{+}-q_{L}^{-}}{q_{L}^{+}+q_{L}^{-}}$
% \end{center}
%
% Na grafe závislosti $\alpha$ od $q_{T}$ vidíme kinematicky povolené oblasti
% $V^{0}$ v~tvare elipsy. Elipsa pre $K^{0}$ prekrýva elipsu pre $\Lambda$ a
% $\overline{\Lambda}$. Graf na obr. 3.2 nám ukazuje zreteľne rozdielne
% $q_{T}$ pre kaóny a lambdy a pomáha nám ich odlíšiť nielen od seba, ale aj
% od  pozadia. Kritérium $q_{T}>0.01\: GeV/c$ nám eliminuje
% značný počet rozpadov $\gamma$ na páry $e^{+}e^{-}$.

Pre jednoznačné $K^{0}$ požadujeme aby $ -0.45< \alpha <+0.45$ pre
jednoznačné $\Lambda$ $\alpha >+0.45$ a pre $\overline{\Lambda}$ $\alpha
<-0.45$.

\begin{center}
  \includegraphics*[bb=75 280 460 690]{armenteros.eps}
\end{center}
\begin{center}
  Obrázok 4.3: Armenteros--Podolanski graf.
\end{center}

Aby $K^{0}$ mali dobrú akceptanciu (o akceptancii pojednám v osobitnej
kapitole) budeme požadovať, aby priečna hybnosť a rapidita ležali v intervale
$1<p_{T}<3\: GeV/c$ a $2.5<y_{lab}<3.3$.   Nakoniec, $V^{0}$ vhodné ako
kandidáti na $K^{0}_{S}$ mezóny sú tie, ktorých invariantná hmotnosť leží v
intervale $\pm 50\: MeV$ vzhľadom k tabuľkovej  hmotnosti $K^{0}_{S}$ mezónu,
to znamená $447 < M_{K^{0}_{S}} < 547\: MeV$.

\subsection{Typy $V^{0}$}
Ako už bolo skôr spomenuté, väčšina $V^{0}$ pozadia pochádza
z~rekonštrukcie dráh, ktoré majú svoj pôvod v~terčíku, ale vyzerajú pri tom
ako $V^{0}$ rozpady. Stopy skutočných $V^{0}$ rozpadových dráh sa
vo všeobecnosti pretínajú dvakrát. Raz vo vrchole a potom opäť v nejakej
vzdialenosti od vrcholu. Ukazuje sa, že maximálna vzdialenosť medzi týmito
pretínajúcimi sa bodmi nezávisí od hybnosti $V^{0}$, ale iba od hybnosti
rozpadajúcich sa častíc v~CMS sústave, $p^{*}$, a od intezity magnetického
poľa, $B$. Z~toho môžeme usúdiť, že pre daný typ $V^{0}$ v~danom magnetickom
poli je maximálna vzdialenosť medzi pretínajúcimi bodmi konštantná.
\begin{center}
  \includegraphics*[bb=150 480 470 740]{impakt.eps}
\end{center}
\begin{center}
  Obrázok 4.4: Dvakrát sa pretínajúce stopy  $V^{0}$ rozpadových dráh.
\end{center}

Obr. 4.4 ukazuje stopy rozpadových dráh pretínajúcich vrchol, ktorý môže byť
v~bode A~alebo v~bode B. Vzdialenosť medzi týmito pretínajúcimi bodmi je
$L$, kde $L$ je definované ako
\begin{center}
  $\sin\theta=\frac{L}{2R}$.
\end{center}
$R$ je polomer krivosti dráhy častice v~magnetickom poli B a jej projekcia na
x-ovú os je $x=L\sin\theta$.

Interakcia medzi nabitými časticami a magnetickým poľom je
\begin{center}
  $e(\overline{\upsilon} \times \overline{B}) = \frac{mv^{2}}{R} \Rightarrow
  R=\frac{p}{e|B|\sin\phi},$
\end{center}
kde $p$ je hybnosť rozpadového produktu a $\phi$ je uhol medzi vektorom
magnetickej indukcie a pohybom nabitej častice v~tomto poli. Pre malé
$\theta$ môžeme písať
\begin{center}
  $\sin\theta \simeq \tan\theta = \frac{q_{T}}{q_{L}}$.
\end{center}
Maximálna vzdialenosť $L$ medzi priesečníkmi bude, ak $q_{T}$ je vo svojom
maxime a pretože $q_{T}=p^{*}\cos\theta^{*}$, môžeme písať $q_{Tmax}=p^{*}$.
Pre rozpad $K^{0}\longrightarrow \pi^{+} \pi^{-}$ je $p^{*}=0.2\: Gev/c$ a
ak rozpadové produkty majú hybnosť niekoľko Gev/c, tak  $p\gg p^{*}
\Rightarrow q_{L} \simeq p$. Predchádzajúcu rovnicu môžeme teraz vyjadriť
v~tvare
\begin{center}
  $\sin\theta_{max}\simeq\frac{p^{*}}{p}$.
\end{center}
Použitím všetkých doterajších vzťahov dostávame vyjadrenie pre maximálnu
vzdialenosť $L_{max}$ rovnú
\begin{center}
  $L_{max}\simeq2(\frac{p}{Be\sin\phi})\times(\frac{p^{*}}{p}) =
  \frac{2p^{*}}{Be\sin\phi}$
\end{center}
a keďže $x=L\sin\phi$, tak dostávame
\begin{center}
  $x=\frac{2p^{*}}{eB}$.
\end{center}
$B = 1.8$ Tesla pre experiment WA94  a teda $x_{max}\simeq 75\: cm$  pre
$K^{0}$.

Ako sme už spomenuli, akceptujeme iba $K^{0}$ kandidátov, ktorí sa
rozpadajú vo vzdialenosti minimálne  115 cm od terčíka. To znamená, že
rozpadové dráhy v~tejto oblasti nemôžu byť spätne extrapolované do
terčíka.

Dráhy  nabitých rozpadových produktov sa tiež môžu alebo nemusia, v
závislosti od geometrie rozpadu, opäť prekrížiť v magnetickom poli, ktoré
tieto dráhy zakrivuje. Dva možné spôsoby rozpadu sú zobrazené na obr. 4.5,
kde rozpad a) je známy pod menom \uv{sailor} a b) ako \uv{cowboy}.

\begin{center}
  \hspace*{-2.5cm}
  \input{pic/cs.pic}
\end{center}
\vspace*{-8cm}
\begin{center}
  Obrázok 4.5: Rozpad $V^{0}$ ako a) \uv{sailor} b)\uv{cowboy}.
\end{center}

Impakt parameter $|\triangle y|$ je definovaný ako vzdialenosť dráhy
rozpadového produktu od centra terčíka v~rovine x-y. Pre pióny ako rozpadové
produkty $K^{0}$ mezónu je tento parameter $|\triangle y|>3cm$, ak majú
pochádzať z~rozpadov $K^{0}$ a nie z~terčíka.

\subsection{Zhrnutie kritérii na identifikáciu $K^{0}_{S}$}
\begin{enumerate}
\item{Každá rozpadová dráha má najmenej štyri priestorové body,}
\item{každá dráha prechádza cez všetkých sedem MWPC,}
\item{vrchol $K^{0}$ leží v intervale $-100<x_{\Omega}<71\: cm$,}
\item{vzdialenosť najmenšieho priblíženia dvoch opačne nabitých
  dráh
  $<1\: cm$,}
\item{uhol medzi rekonštruovaným  vektorom hybnosti $K^{0}$ a spojnicou
  stredu terčíka a
  rozpadového vrcholu je  $<0.75^{\circ}$,}
\item{$ -0.45< \alpha <+0.45$,}
\item{$|\triangle y|>3cm$ pre $\pi^{+}$ a $\pi^{-}$,}
\item{invariantná hmotnosť leží v intervale $447 < M_{K^{0}_{S}} < 547\:
  MeV$,}
\item{$1<p_{T}<3\: GeV/c$ a $2.5<y_{lab}<3.3$.}
\end{enumerate}
